\input{../../resources/assignment-head.tex}

\title{HW4 Proofs and Sets}
\date{Sample Solutions}

\begin{document}
\maketitle
\thispagestyle{fancy}

{\bf Assigned questions}

\begin{enumerate}
   \item Consider the predicate $Pr(x)$ over the set of integers, which evaluates to $T$ exactly when 
   $x$ is prime. Consider the following statements.
   
    \begin{multicols}{2}
    \begin{enumerate}[label=(\roman*)]
        \item $\exists x \in \mathbb{Z}~ \forall y \in \mathbb{Z}~(~x \leq y \to Pr(y)~)$
        \item $\exists x \in \mathbb{Z}~ \forall y \in \mathbb{Z}~(~y \leq x \to Pr(y)~)$
        \item $\forall x \in \mathbb{Z}~ \exists y \in \mathbb{Z}~(~x \leq y \to Pr(y)~)$
        \item $\forall x \in \mathbb{Z}~ \exists y \in \mathbb{Z}~(~y \leq x \to Pr(y)~)$
        \item $\exists x \in \mathbb{Z}~ \forall y \in \mathbb{Z}~(~Pr(y) \to y \leq x~)$
        \item $\exists x \in \mathbb{Z}~ \forall y \in \mathbb{Z}~(~Pr(y) \to x \leq y~)$
        \item $\forall x \in \mathbb{Z}~ \exists y \in \mathbb{Z}~(~Pr(y) \to y \leq x~)$
        \item $\forall x \in \mathbb{Z}~ \exists y \in \mathbb{Z}~(~Pr(y) \to x \leq y~)$
    \end{enumerate}
    \end{multicols}
   
   \begin{enumerate}
   
   \item ({\it Graded for correctness of choice and fair effort completeness in justification
   \footnote{Graded for correctness means your solution will be
   evaluated not only on the correctness of your answers, but on your ability to 
   present your ideas clearly and logically. You should explain how you arrived at your conclusions, using 
   mathematically sound reasoning. Whether you use formal proof techniques or write a more informal argument for why 
   something is true, your answers should always be well-supported. Your goal should be to convince the reader that 
   your results and methods are sound. Graded for fair effort completeness means 
   you will get full credit so long as your submission demonstrates honest 
   effort to answer the question. You will not be penalized for incorrect answers.}}) 
   Which of the statements (i) - (viii) is being {\bf proved} by the following proof:
   \begin{quote}
     Choose $x = 1$, an integer, and we will work to show
     it is a {\bf witness} for the existential claim. By universal generalization, {\bf choose} $e$ to be an {\bf arbitrary} integer. 
     Towards a {\bf direct proof}, {\bf assume} that $Pr(e)$ holds. We {\bf WTS} that $1 \leq e$.
     By definition of the  predicate $Pr$, since $Pr(e)$ is true, $e > 1$. By definition of $\leq$, 
     this means that $1 \leq e$, as required and the claim has been proved. $\square$
   \end{quote}
   
   
   {\it Hint: it may be useful to 
   identify the key words in the proof that indicate proof strategies.}
   
  {\bf Solution}:  The key word ``witness'' indicates proving an existential claim,
  so the options so far are (i) , (ii) , (v), (vi). The next
  sentence proceeds by universal generalization, so the next logical structure
  is that of universal quantification, which is still consistent with each of the 
  four options so far. The next proof strategy used is that of direct proof, where 
  the assumption is that $Pr(e)$. This means we are proving a conditional statement
  whose hypothesis is $Pr(...)$, the assertion that a number is prime (where the number
  is represented by the variable being universally quantified, because $e$ was 
  chosen as the arbitrary element in the universal generalization strategy).
  The only statements that match this are (v) and (vi), and we notice
  that $e$ is replacing $y$ in the predicate. Next, the proof
  works to show $1 \leq e$, where $1$ was our candidate witness choice to replace 
  $x$ and $e$ is the arbitrary element replacing $y$. This means that the goal 
  in the direct proof is $x \leq y$, so the conditional statement originally looked
  like $Pr(y) \to x \leq y$, namely option (vi).

   \item ({\it Graded for correctness of choice and fair effort completeness in justification}) 
   Which of the statements (i) - (viii) is being {\bf disproved} by the following proof:
   \begin{quote}
     To disprove the statement, we will prove the universal
     statement that is logically equivalent to its negation. 
     By universal generalization, {\bf choose} $e$ to be an {\bf arbitrary} integer. 
     We need to find a {\bf witness} integer $y$ such that $y \leq e$ and $\lnot Pr(y)$.
     Notice that $e > 1 \lor e \leq 1$ is true, and we proceed in a {\bf proof by cases}.
     {\bf Case 1}: Assume $e > 1$ and {\bf WTS} there is a witness integer $y$ such that
     $y \leq e$ and $\lnot Pr(y)$. Choose $y = 0$, an integer. Then, since by {\bf case assumption}
     $1 < e$, we have $y = 0 \leq 1 \leq e$.
     Moreover, since $y = 0$, $y > 1$ is false and so (by the definition of $Pr$), the predicate $Pr$
     evaluated at $y$ is false, as required to prove the {\bf conjunction} $y \leq e$ and $\lnot Pr(y)$. 
     {\bf Case 2}: 
     Assume $e \leq 1$ and {\bf WTS} there is a witness integer $y$ such that
     $y \leq e$ and $\lnot Pr(y)$. Choose $y = e-1$, an integer (because subtracting
     $1$ from the integer $e$ still gives an integer). By definition of subtraction, $y = e-1 \leq e$.
     Moreover, since by the {\bf case assumption} $y = e-1 \leq 1-1= 0$, $y > 1$ is false. Thus, 
    (by the definition of $Pr$), the predicate $Pr$
     evaluated at $y$ is false. We have proved the {\bf conjunction} $y \leq e$ and $\lnot Pr(y)$ as required.
     Since each case is complete, the proof by cases is complete and the original
     statement has been disproved.  $\square$
   \end{quote}
   
   {\it Hint: it may be useful to 
   identify the key words in the proof that indicate proof strategies.}

   {\bf Solution}: The original strategy this argument starts with matches our strategy
   for disproving existential claims. We will trace the argument to discover which
   universal claim is being proved, then negate it and use De Morgan to recover the original 
   existential claim being disproved. Proceeding by universal generalization and then 
   finding a witness means the logical structure of the statement being claimed is 
   $\forall ... \exists y$. The goal for the rest of the proof is stated to be ``$y \leq e$ and $\lnot Pr(y)$''
   which would be the negation of the conditional statement $y \leq e \to Pr(y)$.
   This matches the predicate in the statement (ii).

   \item ({\it Graded for correctness of evaluation of statement (is it true or false?)
   and fair effort completeness of the translation and of the proof}) 
    Translate the statement to English and then prove or disprove it
   $$\forall x \in \mathbb{Z}~ \forall y \in \mathbb{Z}~(~x \neq y \to (Pr(x) \lor Pr(y))~)$$

   {\bf Solution}: This statement translates to ``for all integers $x,y$, when $x \neq y$ then 
   at least one of $x$, $y$ is prime''. This statement is {\bf false}, as we can see 
   from the counterexample $x = 1$, $y=4$: these are both integers so they're candidate
   counterexamples, $1 \neq 4$ so the hypothesis of the conditional is true, and
   $Pr(1)$, $Pr(4)$ are both false (as we mentioned in class), so the conclusion of the 
   conditional is false. Namely, the predicate being universally claimed is false at this element 
   of the domain so we have found a counterexample.

   \item ({\it Graded for correctness of evaluation of statement (is it true or false?) 
   and fair effort completeness of the translation and proof}) 
   Translate the statement to English and then prove or disprove it
   $$\left( ~\forall x \in \mathbb{Z} ~Pr(x)~\right) \oplus \left(~\exists x \in \mathbb{Z} ~Pr(x) ~\right)$$

   {\bf Solution}: This statement translates to ``either all integers are prime 
   or at least one integer is prime, and not both''. This statement is {\bf true}, as we can see 
   by proving that exactly one of the universal statements is true. First, notice that 
   $\forall x \in \mathbb{Z} ~Pr(x)$ is false, using the counterexample $x = 1$ (an integer that is not prime).
   Second, notice that $\exists x \in \mathbb{Z} ~Pr(x)$ is true, using the witness $x = 2$ (an integer that is prime).
   Thus, the exclusive or evaluates to true.

   \item ({\it Graded for correctness of evaluation of statement (is it true or false?) 
   and fair effort completeness of the translation and of the proof}) 
    Translate the statement to English and then prove or disprove it
   $$\forall x \in \mathbb{Z}~ \forall y \in \mathbb{Z}~(~(~Pr(x) \land Pr(y)~) \leftrightarrow Pr(x+y)~)$$

   {\bf Solution}: This statement translates to ``the sum of two integers is prime if and only if
   both of them are prime''. This statement is {\bf false}, as we can see 
   from the counterexample $x = 2, y = 2$. These are integer values and evaluating the 
   predicate gives that $Pr(2) \land Pr(2)$ is true (since $2$ is prime so each conjunct is true) while
   $Pr(2+2)$ is false (since $2$ is a positive factor of $4$ that is neither $1$ nor $4$) so the 
   biconditional statement is false.
   

   \item ({\it Graded for correctness of evaluation of statement (is it true or false?)
   and fair effort completeness of the translation and of the proof}) 
    Translate the statement to English and then prove or disprove it
   $$\forall x \in \mathbb{Z}~ (~Pr(x) \to \exists y \in \mathbb{Z}~(~x < y \land Pr(y)~)$$

   {\bf Solution}: This statement translates to ``Every prime is less than another prime''. 
   This statement is {\bf true}. The full proof requires showing that there are infinitely many 
   primes, which we will do in class in a couple of weeks. The idea is that we can always find a 
   number that doesn't have any of the positive integers less than it (other than $1$)
   as a factor.

   \end{enumerate}

   
   \item Let $W = \mathcal{P}(\{1,2,3,4,5\})$. 
   
   \rule{0.5\textwidth}{.4pt}
   
   {\it Sample response that can be used as reference for the detail expected 
   in your answers for this question:} 
   
   To give a witness for the existential claim
   $$ \exists B \in W~( B \in ~\{ X \in W ~|~ 1 \in X \} \cap \{ X \in W ~|~  2 \in X \}~~)$$
   consider $B = \{ 1,2\}$. To prove that this is a valid witness, we need
   to show that it is in the domain of quantification $W$ and that 
   it makes the predicate being quantified evaluate to true. By definition 
   of set-builder notation and intersection, it's enough to prove
   that $\{1,2\} \in W$ and that $1 \in \{1,2\}$ and that $2 \in \{1,2\}$.
   \begin{itemize}
   \item By definition of power set, elements of $W$ are subsets of $\{1,2,3,4,5\}$. Since
   each element in $\{1,2\}$ is an element of $\{1,2,3,4,5\}$, $\{1,2\}$ is a subset of $\{1,2,3,4,5\}$ 
   and hence is an element of $W$. 
   \item Also, by definition of the roster method, $1 \in \{1,2\}$. 
   \item Similarly, by definition of roster method, $2 \in \{1,2\}$.
   \end{itemize}
   Thus $B = \{1,2\}$ is an element of the domain which is in the intersection of the 
   two sets mentioned in the predicate being quantified and is a witness to the existential claim. QED
   
   \rule{0.5\textwidth}{.4pt}
   
   
   \begin{enumerate}
   \item ({\it Graded for correctness}) Give a witness to the existential claim
   $$ \exists X \in W ~(~X \cup X = \emptyset~)$$
   Justify your example by explanations that include references to the relevant definitions.
   
   {\bf Solution}: To give a witness for this existential claim
   consider $X = \emptyset$. To prove that this is a valid witness, we need
   to show that it is in the domain of quantification $W$ and that 
   it makes the predicate being quantified evaluate to true. Since $W$ is a power set of a set
   and we saw in class that the empty set is an element of every power set (because
   it is a subset of every set), $\emptyset \in W$. Now, we calculate $\emptyset \cup \emptyset$.
   By definition of union, $\emptyset \cup \emptyset = \{ x \mid x \in \emptyset \lor x \in \emptyset\} = 
   \{x \mid F \lor F \}  =\{ x | F\} = \emptyset$, as required.

   \item ({\it Graded for correctness}) Give a counterexample to the universal claim
   $$ \forall X \in W ~( \{ a \in X \mid a \textrm{ is even} \} \subsetneq X~)$$
   Justify your example by explanations 
   that include references to the relevant definitions.
   
   {\bf Solution}: To give a counterexample for this universal claim
   consider $X = \{2,4\}$. To prove that this is a valid counterexample, we need
   to show that it is in the domain of quantification $W$ and that 
   it makes the predicate being quantified evaluate to false. Since $2 \in \{1,2,3,4,5\}$
   and $4 \in \{1,2,3,4,5\}$, by definition of subsets, $X \subseteq \{1,2,3,4,5\}$.
   Thus, by definition of the power set, $X \in \mathcal{P}(\{1,2,3,4,5\})$
   so by definition of $W$, $X \in W$. 
   By definition of set builder notation, $\{ a \in X \mid a \textrm{ is even} \}$
   is the subset of $X$ that includes only the elements of $X$ that are even. Since
   $2$ and $4$ are both even, by definition of $X$, $\{ a \in X \mid a \textrm{ is even} \} = X$.
   By definition of proper subsets, $\{ a \in X \mid a \textrm{ is even} \}$ is not a 
   proper subset of $X$, or in other words, $\{ a \in X \mid a \textrm{ is even} \} \subsetneq X$ is 
   false, as required.

   \item  ({\it Graded for correctness}) Give a witness to the existential claim
   $$ \exists (X,Y) \in W \times W ~(~X \cup Y = Y~)$$
   Justify your example by explanations that include references to the relevant definitions.

   {\bf Solution}: A witness will be an ordered pair each of whose components is a subset of $\{1,2,3,4,5\}$
   and which make the property true. Consider $(X,Y) = (\{1,2\}, \{1,2,3\})$. This is an ordered 
   pair and since $1 \in \{1,2,3,4,5\}$, $2 \in \{1,2,3,4,5\}$, and $3 \in \{1,2,3,4,5\}$, 
   $X \subseteq \{1,2,3,4,5\}$ and $Y \subseteq \{1,2,3,4,5\}$ so $(X,Y) \in W \times W$.
   By definition of union, 
   \begin{align*}
    X \cup Y &= \{ a \mid a \in \{1,2\} \lor a \in \{1,2,3\} \} \\
        &=\{ a \mid a = 1 \lor a = 2 \lor a = 1 \lor a = 2 \lor a = 3 \} ~~ \text{by definition of roster method} \\
        &= \{ a \mid a = 1 \lor a = 2 \lor a = 3 \} ~~ \text{since $p \lor p \equiv p$} \\
        & = \{1,2,3\} = Y \qquad \text{by definition of roster method}
   \end{align*}
   \end{enumerate}
   

   \item Recall our representation of movie preferences in a three-movie database 
   using $1$ in a component to indicate liking the movie represented by that component, 
   $-1$ to indicate not liking the movie, and $0$ to indicate neutral opinion or
   haven't seen the movie. We call $Rt$ the set of all ratings $3$-tuples. 
   We defined the function 
   $d_0: Rt\times Rt \to \mathbb{R}$ which takes an ordered pair of ratings $3$-tuples and returns a measure
   of the distance between them 
   given by
   \[
   d_0 (~(~ (x_1, x_2, x_3), (y_1, y_2, y_3) ~) ~) = \sqrt{ (x_1 - y_1)^2 + (x_2 - y_2)^2 + (x_3 -y_3)^2}
   \]
   Another measure of the distance between a pair of ratings $3$-tuples is given by 
   the following function $d_1: Rt\times Rt \to \mathbb{R}$ given by 
   \[
   d_1 (~(~ (x_1, x_2, x_3), (y_1, y_2, y_3) ~) ~) = \sum_{i=1}^3 |x_i - y_i|
   \]
   \begin{enumerate}
    \item  For each of the statements below, first translate them symbolically (using
        quantifiers, logical operators, and arithmetic operations), then determine whether each 
        is true or false by applying the proof strategies to prove each statement or its negation.
        ({\it Graded for correctness of evaluation of statement (is it true or false?) and 
        fair effort completeness of the translation and of the proof}) 
        \begin{enumerate}
            \item For all ordered pairs of ratings $3$-tuples, the value of the function $d_0$ 
            is greater than the value of the function $d_1$.

            {\bf Solution}: We translate the statement to 
            \[
              \forall x \in Rt \times Rt ~(~d_0 (x) > d_1 (x)~)
            \]
            This statement is {\bf false}, as we can see from the counterexample $x = (~(1,0,-1), (1,0,-1)~) \in Rt\times Rt$. 
            Using the definitions of the functions
            \[
              d_0(x) = \sqrt{ (1-1)^2 + (0-0)^2 + (-1--1)^2} = \sqrt{0 + 0 + 0} = 0
            \]
            and
            \[
              d_1(x) = |1-1| + |0-0| + |-1--1| = 0 + 0 + 0 =0
            \]
            Since $d_0(x) = d_1(x)$, it is not the case that $d_0(x) > d_1(x)$.

            \item The maximum value of the function $d_1$ is greater than the maximum value of the function $d_0$.
            
            {\bf Solution}: We translate the statement to 
            \[  
            \exists x_0 ~\exists x_1  ~(~\forall y (~d_0(x_0) \geq d_0(y)~) \land \forall y (~d_1(x_1)\geq d_1(y)~) \land d_1(x_1) > d_0(x_0)~)
            \]
            (where all the quantifiers are over the domain $R_t \times R_t$). 
            Notice that the logical structure of this statement is
            ``there are two values where the first is the input to $d_0$ that gives the maximum value of $d_0$, 
            the second is the input to $d_1$ that gives the maximum value of $d_1$, and 
            this maximum value of $d_1$ is greater than or equal to this maximum value of $d_0$".
            In particuar, we used quantifiers to define how to calculate maximum values. 
            
            This statement is true
            because the maximum values for each of $d_1$ and $d_0$ are achieved when the inputs to the function 
            have opposite ratings in each component and we can calculate the maximum for $d_1$ in this case will 
            be $6$ whereas for $d_0$ it will be $\sqrt{12} = 2 \sqrt{3} < 6$ (because $\sqrt{3} < 3$). 
            
            With more details: 
            we will show that $x_0 = x_1 = (~(1,1,1), (-1,-1,-1)~)$ will witness the existential claim.
            To do so, we have three goals:
            \begin{itemize}
              \item Goal 1: we need to show that $\forall y (~d_0(x_0) \geq d_0(y))$
              \item Goal 2: we need to show that $\forall y (~d_1(x_1) \geq d_1(y))$
              \item Goal 3: we need to show that $d_1(x_1) > d_0 (x_0)$
            \end{itemize}
            To work towards these goals, it will be useful to calculate 
            \[
              d_1(x_1) = |1--1| + |1--1| + |1--1| = 2 + 2 + 2 = 6 = 2 \cdot 3
            \]
            \[
              d_0(x_0) = \sqrt{(1--1)^2 + (1--1)^2 + (1--1)^2} = \sqrt{4 + 4+ 4} = \sqrt{12} = 2\sqrt{3}
            \]
            Since $\sqrt{3} < 3$, these calculations prove that $d_1(x_1) > d_0 (x_0)$ and thus Goal 3 has been shown.
            To prove Goals (1) and (2), we notice that, since each component of a ratings 3-tuple must be 
            $-1$, $0$, or $1$, the maximum absolute difference between component values is $2$. Moreover, since these
            difference are added up (in $d_1$) or squared and then added up (in $d_0$), the function value for any ordered
            pair of ratings $3$-tuples will be less than or equal to the values we calculated for $x_0$ and $x_1$. 
        \end{enumerate}

    \item ({\it Graded for correctness}) Write a statement about 3-tuples of movie ratings that uses the function 
    $d_1$ and has at least one universal and one existential quantifier. Your response will be 
    graded correct if all the syntax in your statement is correct.

    {\bf Solution}: Consider the statement $\forall x \in Rt ~\exists y \in Rt ~(~d_1(~(x,y)~) = 0~)$. 

    \item ({\it Graded for fair effort completeness}) Translate the property you wrote symbolically in the 
    last step to English. Indicate if it is true, false, or if you don't know 
    (sometimes we can write interesting properties, and we're not sure if they are true or not!). 
    Give informal justification for whether  you think it is true/ false, or explain why 
    the proof strategies we have so far do not appear to  be sufficient to determine whether the statement holds.

    {\bf Solution}: The statement we wrote translates to ``for each ratings 3-tuple there is some ratings 3-tuple 
    that is $d1$ distance zero from it.'' This statement is true because, for arbitrary ratings 3-tuple $x$
    we can choose the witness ratings 3-tuple $y=x$ and then use the definition of $d_1$ to calculate that the 
    output of the function will be zero since $y=x$.
    \end{enumerate}

\end{enumerate}
\end{document}