\input{../../resources/assignment-head.tex}

\title{HW3 Circuits and Logic}
\date{Due: Tuesday, October 26, 2021 at 11:00PM on Gradescope}

\begin{document}
\maketitle
\thispagestyle{fancy}

{\bf In this assignment,}

You will consider how circuits and logic can be used to represent
mathematical and technical claims. You will use propositional 
and predicate logic to evaluate these claims.

Instructions and academic integrity reminders for all homework assignments in 
CSE20 this quarter are on the class website and on the hw1-definitions-and-notations
assignment.

You will submit this assignment via Gradescope
(\href{https://www.gradescope.com}{https://www.gradescope.com}) 
in the assignment called ``hw3-circuits-and-logic''.

{\bf Resources}: To review the topics you are working with 
for this assignment, see the class material from Week 3 and Week 4.
We will post frequently asked questions and our answers to them in a 
pinned Piazza post.

{\bf Assigned questions}

\begin{enumerate}

   \item Imagine a friend suggests the following argument to you: ``The compound proposition
   \[
   (x \to y) \to z
   \]
   is logically equivalent to 
   \[
   x \to (y \to z)
   \]
   because I can transform one to the other using the following sequence of logical equivalences
   like distributivity and associativity and using that $(p \to q) \equiv (\lnot p \lor q)$: 
   \[
      (x \to y) \to z \equiv
      \lnot (\lnot x \lor y) \lor z \equiv
      \lnot x \lor (\lnot y \lor z) \equiv
      x \to (\lnot y \lor z) \equiv x \to (y \to z)
   \]
   so conditionals
   are associative just like conjunctions and disjunctions".
   
   \begin{enumerate}
   \item ({\it Graded for correctness}\footnote{This means your solution will be
   evaluated not only on the correctness of your answers, but on your ability to 
   present your ideas clearly and logically. You should explain how you arrived at your conclusions, using 
   mathematically sound reasoning. Whether you use formal proof techniques or write a more informal argument for why 
   something is true, your answers should always be well-supported. Your goal should be to convince the reader that 
   your results and methods are sound.}) Prove to your friend that they made a mistake by giving a truth
   assignment to the propositional variables where 
   the two compound propositions 
   $ (x \to y) \to z$ and $ x \to (y \to z)$ have different truth values.
   Justify your choice by evaluating these compound propositions using the definitions of the logical connectives 
   and include enough intermediate steps so that a student in CSE 20 who may be 
   struggling with the material can still follow along with your reasoning.
   
   \item ({\it Graded for fair effort completeness}\footnote{This means you will get full credit so long as your submission 
   demonstrates honest effort to answer the question. You will not be penalized for incorrect answers.}) 
   Help your friend find the problem in their argument by pointing out which step(s) were incorrect.
   
   \item ({\it Graded for fair effort completeness}) Give {\bf three} different compound propositions
   that are actually logically equivalent to (and not the same as)
   \[
   x \to (y \to z)
   \]
   Justify each one of these logical equivalences either by applying a sequence of logical equivalences
   or using a truth table.  Notice that you can use other logical operators (e.g. $\lnot, \lor, \land, \oplus, \to, 
   \leftrightarrow$) 
   when constructing your compound propositions.

   {\it Bonus; not for credit (do not hand in)}: How would you translate each of the equivalent compound
   propositions in English? Does doing so help illustrate why they are equivalent?
   \end{enumerate}
   
   \item For each part of this question you will use the following input-output definition table 
   with four inputs $x_3$, $x_2$, $x_1$, $x_0$
   
   \begin{center}
   \begin{tabular}{cccc|c}
   $x_3$ & $x_2$ & $x_1$ & $x_0$ & $out$\\
   \hline
   $1$ & $1$ & $1$ & $1$ & $1$\\
   $1$ & $1$ & $1$ & $0$ & $0$\\
   $1$ & $1$ & $0$ & $1$ & $0$\\
   $1$ & $1$ & $0$ & $0$ & $0$\\
   $1$ & $0$ & $1$ & $1$ & $0$\\
   $1$ & $0$ & $1$ & $0$ & $1$\\
   $1$ & $0$ & $0$ & $1$ & $0$\\
   $1$ & $0$ & $0$ & $0$ & $0$\\
   $0$ & $1$ & $1$ & $1$ & $0$\\
   $0$ & $1$ & $1$ & $0$ & $0$\\
   $0$ & $1$ & $0$ & $1$ & $1$\\
   $0$ & $1$ & $0$ & $0$ & $0$\\
   $0$ & $0$ & $1$ & $1$ & $0$\\
   $0$ & $0$ & $1$ & $0$ & $0$\\
   $0$ & $0$ & $0$ & $1$ & $0$\\
   $0$ & $0$ & $0$ & $0$ & $1$\\
   \end{tabular}
   \end{center}
   \begin{enumerate}   
   \item  ({\it Graded for fair effort completeness}) 
   Construct a compound proposition that implements this input-output table
   and draw a combinatorial circuit corresponding to this compound proposition. We recommend
   the following steps:
   
   \begin{itemize}
   \item Draw symbols for the inputs on the left-hand-side and for the output on the right-hand side.
   \item Construct an expression for $out$ using (some of) the inputs 
   $x_3, x_2, x_1, x_0$ and the logic gates XOR, AND, OR, NOT. {\it Hint:} are normal forms helpful here?
   How do you choose which normal form to use?
   \item Draw and label the gates corresponding to the expression you construct, and connect appropriately with wires.
   \end{itemize}
   
   \item ({\it Graded for correctness}) 
   For each of the following predicates, consider whether this input-output table (and the
   logic circuit from part (a)) implements the rule for 
   the predicate.
   If yes, explain why using the definitions of the operations involved and consider {\bf all} possible 
   inputs. If no, explain why not by providing specific example input values 
   and using the input-output definition table to compute the logic circuit's output for this example input
   and then comparing with the value of the predicate in question to justify your example.

   \begin{enumerate}
      \item $P_1: \{0,1\}\times \{0,1\} \times \{0,1\} \times \{0,1\} \to \{T, F\}$ given by 
      $$P_1(~(x_3,x_2,x_1,x_0)~) = \begin{cases}  T \quad &\text{when~}(x_3x_2x_1x_0)_{2,4} \leq 5 \\ F &\text{otherwise}\end{cases}$$
      \item $P_2: \{0,1\}\times \{0,1\} \times \{0,1\} \times \{0,1\} \to \{T, F\}$ given by 
      $$P_2(~(x_3,x_2,x_1,x_0)~) = \begin{cases}  T \quad &\text{when~} (x_3x_2x_1x_0)_{2,4} \textbf{ mod } 5 = 0 \\  F &\text{otherwise}\end{cases}$$
      \item $P_3: \{0,1\}\times \{0,1\} \times \{0,1\} \times \{0,1\} \to \{T, F\}$ given by 
      $$P_3(~(x_3,x_2,x_1,x_0)~) = \begin{cases}  T \quad &\text{when~} (x_3x_2x_1x_0)_{2,4}  = [x_3x_2x_1x_0]_{s,4} \\  F &\text{otherwise}\end{cases}$$
   \end{enumerate}
   \end{enumerate}

   \item Recall the functions \textit{mutation}, \textit{insertion}, and \textit{deletion} defined in class.
   We define the predicates:

   $Mut$ with domain $S \times S$ is defined by, for $s_1 \in S$ and $s_2 \in S$,
   \[
      Mut(~(s_1,s_2)~) = \exists k\in \mathbb{Z^+} \exists b \in B (~ mutation(~(s_1, k, b)~) = s_2~)
   \]
   $Ins$ with domain $S \times S$ is defined by, for $s_1 \in S$ and $s_2 \in S$,
   \[
      Ins(~(s_1,s_2)~) = \exists k\in \mathbb{Z^+} \exists b \in B (~ insertion(~(s_1, k, b)~) = s_2~)
   \]
   $Del$ with domain $\{ s\in S \mid rnalen(s) > 1\}  \times S$ is defined by, for 
   $s_1 \in \{ s\in S \mid rnalen(s) > 1\} $ and $s_2 \in S$,
   \[
      Del(~(s_1,s_2)~) = \exists k\in \mathbb{Z^+} (~ deletion(~(s_1, k)~) = s_2~)
   \]

   For each quantified statement below, {\bf first} translate to an English sentence.

   {\bf Then}, negate the {\bf whole} statement and rewrite this
   negated statement so that negations appear only within predicates 
   (that is, so that no negation is outside a quantifier or an expression involving logical connectives).
 
    The translations are graded for fair effort completeness.
 
   The negations are graded for correctness. For negations: You do not need to justify 
   your work for this part of the question.  However, if you include correct
   intermediate steps, we might be able to award partial credit for an incorrect answer.
 
  \rule{0.5\textwidth}{.4pt}
 
 {\it Sample response that can be used as reference for the detail expected 
 in your answer:} 
 Consider the statement
 \[
  \hspace{-1in}\forall n \in \mathbb{Z}^+~ \exists s \in S~\left( ~L(~(s,n)~) \land F_\A (s) \land BC(~(s,\A,n)~)~\right)
  \]
  
 {\bf Solution}: English translation is 
 \begin{quote}
 For each positive integer there is some RNA strand of that length that starts 
 with \A~ and all of its bases are \A.
 \end{quote} 
 
 We obtain the negation using multiple applications of De Morgan's rule and logical equivalences. 
 \begin{align*}
 \lnot &\forall n \in \mathbb{Z}^+~ \exists s \in S~\left( ~L(~(s,n)~) \land F_\A (s) \land BC(~(s,\A,n)~)~\right) \\
 \equiv&\exists n \in \mathbb{Z}^+~ \lnot \exists s \in S~\left( ~L(~(s,n)~) \land F_\A (s) \land BC(~(s,\A,n)~)~\right) \\
 \equiv&\exists n \in \mathbb{Z}^+~ \forall s \in S~\lnot \left( ~L(~(s,n)~) \land F_\A (s) \land BC(~(s,\A,n)~)~\right) \\
 \equiv&\exists n \in \mathbb{Z}^+~ \forall s \in S~\left( ~\lnot L(~(s,n)~) \lor \lnot F_\A (s) \lor \lnot BC(~(s,\A,n)~)~\right)
 \end{align*}
 
 
 \rule{0.5\textwidth}{.4pt}
 
 
 \begin{enumerate}
 \item  First statement:
 \[
 \forall s \in S ~\left( ~Mut(~(s,s)~) \leftrightarrow Ins(~(s,s)~) ~\right)
 \]
 
 \item Second statement
 \[
 \forall s_1 \in S ~ \forall s_2 \in S ~\forall s_3 \in S ~\left( ~\left(~Mut(~(s_1,s_2)~) \land Mut(~(s_2, s_3)~) ~\right) \to Mut(~(s_1,s_3)~)~\right)
 \]
 
 \item Third statement: we use $S'$ to abbreviate $\{ s\in S \mid rnalen(s) > 1\}$
 \[
 \forall s_2 \in S' ~\exists s_1 \in S'~\left(~ Del(~(s_1,s_2)~)~\right) 
 \land \lnot \forall s_1 \in S' ~\exists s_2 \in S'~\left(~ Del(~(s_1,s_2)~)~\right) 
 \]
 \end{enumerate}

 {\it Bonus; not for credit (do not hand in)}:  For each statement above, is the statement or its negation true? How do you know?
 
 
\end{enumerate}
\end{document}