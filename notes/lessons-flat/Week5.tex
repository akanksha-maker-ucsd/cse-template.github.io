\documentclass[12pt, oneside]{article}

\usepackage[letterpaper, scale=0.89, centering]{geometry}
\usepackage{fancyhdr}
\setlength{\parindent}{0em}
\setlength{\parskip}{1em}

\pagestyle{fancy}
\fancyhf{}
\renewcommand{\headrulewidth}{0pt}
\rfoot{\href{https://creativecommons.org/licenses/by-nc-sa/2.0/}{CC BY-NC-SA 2.0} Version \today~(\thepage)}

\usepackage{amssymb,amsmath,pifont,amsfonts,comment,enumerate,enumitem}
\usepackage{currfile,xstring,hyperref,tabularx,graphicx,wasysym}
\usepackage[labelformat=empty]{caption}
\usepackage[dvipsnames,table]{xcolor}
\usepackage{multicol,multirow,array,listings,tabularx,lastpage,textcomp,booktabs}

\lstnewenvironment{algorithm}[1][] {   
    \lstset{ mathescape=true,
        frame=tB,
        numbers=left, 
        numberstyle=\tiny,
        basicstyle=\rmfamily\scriptsize, 
        keywordstyle=\color{black}\bfseries,
        keywords={,procedure, div, for, to, input, output, return, datatype, function, in, if, else, foreach, while, begin, end, }
        numbers=left,
        xleftmargin=.04\textwidth,
        #1
    }
}
{}
\lstnewenvironment{java}[1][]
{   
    \lstset{
        language=java,
        mathescape=true,
        frame=tB,
        numbers=left, 
        numberstyle=\tiny,
        basicstyle=\ttfamily\scriptsize, 
        keywordstyle=\color{black}\bfseries,
        keywords={, int, double, for, return, if, else, while, }
        numbers=left,
        xleftmargin=.04\textwidth,
        #1
    }
}
{}

\newcommand\abs[1]{\lvert~#1~\rvert}
\newcommand{\st}{\mid}

\newcommand{\A}[0]{\texttt{A}}
\newcommand{\C}[0]{\texttt{C}}
\newcommand{\G}[0]{\texttt{G}}
\newcommand{\U}[0]{\texttt{U}}

\newcommand{\cmark}{\ding{51}}
\newcommand{\xmark}{\ding{55}}

 
\begin{document}
\begin{flushright}
    \StrBefore{\currfilename}{.}
\end{flushright} 
\section*{Monday October 25}
\subsection*{Proof strategies}


We now have propositional and predicate logic that can help us express 
statements about any domain. We will develop proof strategies to 
craft valid argument for proving that such statements are true or disproving
them (by showing they are false). We will practice these strategies with 
statements about sets and numbers, both because they are familiar and because they
can be used to build cryptographic systems. Then we will apply proof strategies
more broadly to prove statements about data structures and machine learning 
applications. 

When a predicate $P(x)$ is over a {\bf finite} domain:
\begin{itemize}
\item To show that $\forall x  P(x)$ is true: check that $P(x)$ evaluates to $T$ at each domain element by evaluating over and over.
\item To show that $\forall x  P(x)$ is false: find one counterexample, a domain element where $P(x)$ evaluates to $F$.
\item To show that $\exists x  P(x)$ is true: find one witness, a domain element where $P(x)$ evaluates to $T$.
\item To show that $\exists x  P(x)$ is false: check that $P(x)$ evaluates to $F$ at each domain element by evaluating over and over.
\end{itemize} 

\fbox{\parbox{\linewidth}{New! {\bf Proof of universal by exhaustion}: To prove that $\forall x \, P(x)$
is true when $P$ has a finite domain, evaluate the predicate at {\bf each} domain element to confirm that it is always T.
}} 

\fbox{\parbox{\linewidth}{

{\bf New! Proof by universal generalization}: To prove that $\forall x \, P(x)$
is true, we can take an arbitrary element $e$ from the domain of 
quantification and show that $P(e)$ is true, without making any assumptions about $e$ 
other than that it comes from the domain.


An {\bf arbitrary} element of a set or domain is a fixed but unknown element from that set. 
}}
 \newpage


{\bf Definitions}:

A {\bf set} is an  unordered collection of  elements.
When $A$ and  $B$ are sets,  $A = B$ (set equality) means  
\[
    \forall x  ( x\in A \leftrightarrow x \in B)
\]

When $A$ and  $B$ are sets, $A \subseteq B$ (``$A$ is a {\bf subset} of $B$") means 
\[
    \forall x  (x \in A  \to x  \in B)
\]

When $A$ and  $B$ are sets,  $A \subsetneq B$ (``$A$ is a {\bf proper subset} of $B$") means 
\[
    (A\subseteq B) \wedge  (A \neq B)
\]

To prove that one set is a subset of another \ldots

\vspace{50pt}

To prove that two sets are equal \ldots

\vspace{50pt}
 

Example: $\{ 43, 7, 9 \} = \{ 7, 43, 9, 7\}$

\vspace{50pt}
 

\fbox{\parbox{\linewidth}{

{\bf New! Proof of conditional by direct proof}: To prove that the conditional statement $p \to q$ is true, 
we can assume $p$ is true and use that assumption to show $q$ is true.
}}

\fbox{\parbox{\linewidth}{

{\bf New! Proof of conditional by contrapositive proof}: To prove that the implication $p \to q$ is true, 
we can assume $q$ is false and use that assumption to show $p$ is also false.
}}
 

\fbox{\parbox{\linewidth}{{\bf New! Proof by Cases}: To prove $q$, we can 
work by cases by first describing all possible cases we might be in
and then showing that each one guarantees $q$.
Formally, if we know that $p_1 \lor p_2$ is true, 
and we can show that $(p_1 \to q)$ is true and we can show that $(p_2 \to q)$, 
then we can conclude $q$ is true.
}} 

\fbox{\parbox{\linewidth}{
{\bf New! Proof of conjunctions with subgoals}:
To show that $p \land q$ is true, we have two subgoals: subgoal (1) prove $p$ 
is  true; and, subgoal (2) prove $q$ is true.

\vspace{1em}

 To show that $p \land q$ is false, it's enough to prove that $\lnot p$.
 
 To show that $p \land q$ is false, it's enough to prove that $\lnot q$.
}} \newpage


{\bf Prove} or {\bf  disprove}: $\{ \A,  \C,  \U,  \G\} \subseteq \{ \A\A, \A\C, \A\U, \A\G \}$ 

\vspace{80pt}

{\bf Prove} or {\bf  disprove}: For some set $B$, $\emptyset \in B$.

\vspace{80pt}

{\bf Prove} or {\bf  disprove}: For every set $B$, $\emptyset \in B$.

\vspace{80pt}

{\bf Prove} or {\bf  disprove}: The empty set is a subset of every set.

\vspace{80pt}

{\bf Prove} or {\bf  disprove}: The empty set is a proper subset of every set.

\vspace{80pt}

{\bf Prove} or {\bf  disprove}: $\{ 4, 6 \} \subseteq \{ n \mid  \exists c \in \mathbb{Z} ( n = 4c) \} $

\vspace{80pt}

{\bf Prove} or {\bf  disprove}: $\{ 4, 6 \} \subseteq \{ n ~\textbf{mod}~10 \mid  \exists c \in \mathbb{Z} ( n = 4c) \} $

\vspace{80pt}

 \newpage
\subsection*{Review}
\begin{enumerate}
\item \hspace{1in}\\ 

Suppose $P(x)$ is  a predicate over a domain $D$.
\begin{enumerate}
    \item True or False: To translate the statement
    ``There are at least two  elements in $D$
    where the predicate $P$ evaluates to true", we
    could  write
    \[
    \exists  x_1 \in D \, \exists x_2 \in D  \, (P(x_1) \wedge P(x_2))
    \]
    \item True or False: To translate the statement
    ``There are at most two  elements in $D$
    where the predicate $P$ evaluates to true", we
    could write
    \[
    \forall  x_1 \in D \, \forall x_2 \in D \, \forall x_3 \in  D \, \left(~ (~P(x_1) \wedge P(x_2)  \wedge P(x_3) ~) \to (~ x_1 = x_2 \vee x_2 = x_3 \vee x_1 = x_3~)~\right)
    \]
\end{enumerate} \item \hspace{1in}\\ 

For each of the following English statements, select
the correct translation, or select None.

{\it Challenge: determine which of the statements are true and which 
are false.}

\begin{enumerate}
\item Every set is a subset of itself.

\item Every set is an element of itself.

\item Some set is an element of all sets.

\item Some set is a subset of all sets.
\end{enumerate}

\begin{enumerate}
\item[i.] $\forall X ~\exists Y ~(X \in Y)$
\item[ii.] $\exists X ~\forall Y ~(X \in Y)$
\item[iii.] $\forall X ~\exists Y ~(X \subseteq Y)$
\item[iv.] $\exists X ~\forall Y ~(X \subseteq Y)$
\item[v.] $\forall X ~(X \in X)$
\item[vi.] $\forall X ~(X \subseteq X)$ 
\end{enumerate} \item We want to hear how the term and this class are going for you.
Please complete the midquarter feedback form: \href{https://forms.gle/w3D7ifAWnD5sWwHf9}{https://forms.gle/w3D7ifAWnD5sWwHf9}
\end{enumerate}

\newpage
\section*{Wednesday October 27}


{\bf Cartesian product}: When $A$ and  $B$ are sets, 
\[
    A \times  B = \{ (a,b) \mid a \in A  \wedge b  \in B \}
\]

Example: $\{43, 9\} \times  \{9, \mathbb{Z}\}  = $
    
Example: $\mathbb{Z} \times \emptyset  = $

{\bf Union}: When $A$ and  $B$ are sets,
\[
    A \cup  B = \{ x \mid x \in A  \vee x \in B \}
\]    
    
Example: $\{43, 9\} \cup \{9, \mathbb{Z}\}  = $

Example: $\mathbb{Z} \cup \emptyset  = $ 

{\bf Intersection}: When $A$ and  $B$ are sets,
\[
    A \cap  B = \{ x \mid x \in A  \wedge x \in B \}
\]    
Example: $\{43, 9\} \cap \{9,\mathbb{Z}\}  = $

Example: $\mathbb{Z} \cap \emptyset  = $


{\bf Set  difference}: When $A$ and  $B$ are sets,

\[
    A -  B = \{ x \mid x \in A  \wedge x \notin B \}
\]

Example: $\{43, 9\} - \{9, \mathbb{Z}\}  = $

Example: $\mathbb{Z} - \emptyset  = $

    
{\bf Disjoint sets}: sets $A$ and  $B$ are disjoint means $A \cap  B  = \emptyset$

Example: $\{43, 9\}, \{9, \mathbb{Z}\}$ are not  disjoint 

Example: The sets $\mathbb{Z}$ and $\emptyset$ are disjoint

    

{\bf Power set}: When $U$ is a set, $\mathcal{P}(U) = \{ X \mid X \subseteq U\}$

Example: $\mathcal{P}(\{43, 9\}) = $

Example: $\mathcal{P}(\emptyset) = $
 \newpage


Let $W =  \mathcal{P}(  \{ 1,2,3,4,5\} )$


Example elements in $W$ are:
\vspace{20pt}

{\bf Prove} or {\bf  disprove}:  $\forall  A \in W\,  \forall B \in W\,  \left( A \subseteq B
~\to ~ \mathcal{P}(A) \subseteq \mathcal{P}(B) \right)$

\vfill
\vfill
\vfill

{\it Extra example}: {\bf Prove} or {\bf  disprove}:  $\forall  A \in W\,  \forall B \in W\,  \left( \mathcal{P}(A)  =\mathcal{P}(B)
~\to ~ A = B \right)$

\vspace{20pt}

{\it Extra example}: {\bf Prove} or {\bf  disprove}:  $\forall  A \in W\,  \forall B \in W\, \forall C  \in W\,  \left( A\cup B  = A \cup  C
~\to ~ B = C \right)$

\vspace{20pt} 


{\bf Assume} \ldots, we {\bf want to show} that \ldots.

\vspace{20pt}

Let \ldots be an {\bf arbitrary} \ldots.  \newpage
\subsection*{Review}
\begin{enumerate}
\item \hspace{1in}\\ 

Let $W =  \mathcal{P}(\{1,2,3,4,5\})$.
The statement $$\forall A \in W~ \forall B\in W~ \forall  C  \in W~  ( A \cup B =  A \cup C ~\to~  B = C) $$ is false.
Which of the following  choices for  $A, B, C$ could  
be used to  give a counterexample to this claim?
(Select all and only that  apply.)
\begin{enumerate}
    \item $A = \{ 1, 2, 3 \}, B = \{ 1, 2\}, C= \{1, 3\}$
    \item $A = \{ 1, 2, 3 \}, B = \{ 2\}, C= \{2\}$
    \item $A = \{ \emptyset, 1, 2, 3 \}, B = \{ 1, 2\}, C= \{1, 3\}$
    \item $A = \{ 1, 2, 3 \}, B = \{ 1, 2\}, C= \{1, 4\}$
    \item $A = \{ 1, 2 \}, B = \{ 2, 3\}, C= \{1, 3\}$
    \item $A = \{ 1,2 \}, B =  \{ 1,3\}, C =  \{ 1,3\}$
\end{enumerate} \item \hspace{1in}\\ 

Let $W =  \mathcal{P}(\{1,2,3,4,5\})$.
Consider the  statement
$$\forall A \in W~ \forall B\in W~  \big( ( \mathcal{P}(A) = \mathcal{P}(B) )~\to~ (A = B) \big) $$

This statement is true. A proof of this statement starts with universal generalization, c
onsidering
arbitrary $A$ and $B$ in $W$. At this point, it remains to prove that 
$( \mathcal{P}(A) = \mathcal{P}(B) )~\to~ (A = B)$
is true about these arbitrary elements.  There are two ways to proceed: 

\begin{itemize}
\item[] First approach: By direct proof, in which we assume the hypothesis of the 
conditional and work to show that the conclusion follows.
\item[] Second approach: By proving the contrapositive version of the conditional instead, in which we
assume the negation of the conclusion and work to show that the negation of hypothesis follows.
\end{itemize} 

Pick an option from below for the assumption and ``need to show" in each approach.

\begin{enumerate}
    \item First approach, assumption.
    \item First approach, ``need to show".
    \item Second approach, assumption.
    \item Second approach, ``need to show".
\end{enumerate}


\begin{multicols}{2}
\begin{enumerate}[label=(\roman*)]
\item $\forall X ( X \subseteq A \leftrightarrow X \subseteq B)$
\item $\exists X ( X \subseteq A \leftrightarrow X \subseteq B)$
\item $\forall X ( X \subseteq A \oplus X \subseteq B)$
\item $\exists X ( X \subseteq A \oplus X \subseteq B)$
\item $\forall x ( x \in A \leftrightarrow x \in B)$
\item $\exists x ( x \in A \leftrightarrow x \in B)$
\item $\forall x ( x \in A \oplus x \in B)$
\item $\exists x ( x \in A \oplus x \in B)$
\end{enumerate}
\end{multicols} \end{enumerate}

\newpage
\section*{Friday October 29}
\subsection*{Facts about numbers}


\begin{enumerate}
    \item Addition and multiplication of real 
    numbers are each commutative and associative. 
    \vspace{25pt}
    \item The product of two positive numbers is positive, of 
    two negative numbers is positive, and of a positive and a negative number is negative.
    \vspace{25pt}
    \item The sum of two integers, the product of two integers, and the 
    difference between two integers are each integers.
    \vspace{25pt}
    \item For every integer $x$ there is no integer strictly between $x$ and $x+1$, 
    \vspace{25pt}
    \item When $a, b$ are positive integers, $ab \geq a$ and $ab \geq b$.
    \vspace{25pt}
\end{enumerate}
 \subsection*{Factoring}


{\bf Definition}: When $a$ and $b$ are integers and $a$ is nonzero, 
{\bf $a$ divides $b$} means there is an integer $c$ such that $b = ac$ . 


Symbolically, $F(~(a,b)~) = \phantom{\exists c\in \mathbb{Z}~(b=ac)}$
and is  a predicate over the domain \underline{\phantom{$\mathbb{Z}^{\neq 0} \times \mathbb{Z}$}}


Other (synonymous) ways to say that $F(~(a,b)~)$ is true: 
\begin{center}
$a$ is a {\bf factor} of $b$
\qquad 
$a$ is a {\bf divisor} of $b$
\qquad  $b$ is a {\bf multiple} of $a$
\qquad
$a | b$
\end{center}

When $a$ is a positive integer and $b$ is any integer, $a | b$
exactly when $b \textbf{ mod } a = 0$

When $a$ is a positive integer and $b$ is any integer, $a | b$
exactly $b = a \cdot (b \textbf{ div } a)$ 

{\it Translate these quantified statements by matching to English statement on right.}

\begin{multicols}{2}
$\exists a\in \mathbb{Z}^{\neq 0} ~(~F(a,a)~)$

$\exists a\in \mathbb{Z}^{\neq 0} ~(~\lnot F(a,a)~)$

$\forall a\in \mathbb{Z}^{\neq 0} ~(~F(a,a)~)$

$\forall a\in \mathbb{Z}^{\neq 0} ~(~\lnot F(a,a)~)$


Every nonzero integer is a factor of itself.

No nonzero integer is a factor of itself.

At least one nonzero integer is a factor of itself.

Some nonzero integer is not a factor of itself.
\end{multicols} 

{\bf Claim}: Every nonzero integer is a factor of itself.

{\bf Proof}: 


\vspace{150pt}


{\bf Prove} or {\bf Disprove}: There is a nonzero integer that does not divide its square.



\vspace{150pt}

{\bf Prove} or {\bf Disprove}: Every positive factor of a positive integer is less than or equal to it.

\vspace{150pt}
 \newpage


{\bf Claim}: Every nonzero integer is a factor of itself and 
every nonzero integer divides its square.

\vspace{250pt}
 

{\bf Definition}:  An integer $p$ greater than $1$ is called {\bf prime} means 
the only positive factors of 
$p$ are $1$ and $p$. A positive integer that is greater than $1$ and is not prime 
is called composite. \\

A formal definition of the predicate $Pr$ over the domain $\mathbb{Z}$ 
which evaluates to T exactly when evaluated on a prime integer is:
\[
    (x > 1) \land \forall a( ~ (~ a > 0 \land F(a,x) ~) \to (a = 1 \lor a = x) ~)
\]
 

{\bf True or False}: The statement ``There are three consecutive positive integers that are prime."

{\it Hint}: These numbers would be of the form $p, p+1, p+2$ (where $p$ is a positive integer).

{\bf Proof}: We need to show \underline{\phantom{$\exists p \in \mathbb{Z}^+ ~(~Pr(p) \land Pr(p+1) \land Pr(p+2)~)$}}

\vfill

{\bf True or False}: The statement ``There are three consecutive odd positive integers that are prime."

{\it Hint}: These numbers would be of the form $p, p+2, p+4$ (where $p$ is an odd positive integer).

{\bf Proof}: We need to show \underline{\phantom{$\exists p \in \mathbb{Z}^+ ~(~(p \textbf{ mod } 2 = 1 \land Pr(p) \land Pr(p+2) \land Pr(p+4)~)$}}

\vfill
 \newpage
\subsection*{Review}
\begin{enumerate}
    \item \hspace{1in}\\ 

Recall the predicate  $F(~(a,b)~)  = ``a \text{ is a factor of } b"$ over  
the domain $\mathbb{Z}^{\neq 0} \times \mathbb{Z}$ we worked with 
in class. Consider the following quantified
statements
\begin{multicols}{2}
\begin{enumerate}[label=(\roman*)]
\item $\forall x \in \mathbb{Z} ~(F(1,x))$
\item $\forall x \in \mathbb{Z}^{\neq 0} ~(F(x,1))$
\item $\exists x \in \mathbb{Z} ~(F(1,x))$
\item $\exists x \in \mathbb{Z}^{\neq 0} ~(F(x,1))$
\item $\forall x \in \mathbb{Z}^{\neq 0} ~\exists y \in \mathbb{Z} ~(F(x,y))$
\item $\exists x \in \mathbb{Z}^{\neq 0} ~\forall y \in \mathbb{Z} ~(F(x,y))$
\item $\forall y \in \mathbb{Z} ~\exists x \in \mathbb{Z}^{\neq 0} ~(F(x,y))$
\item $\exists y \in \mathbb{Z} ~\forall x \in \mathbb{Z}^{\neq 0} ~(F(x,y))$
\end{enumerate}
\end{multicols}
\begin{enumerate}
\item Select the statement whose translation is
\begin{quote}
``The number $1$ is  a factor of every integer."
\end{quote}
or write NONE if none of (i)-(viii) work.

\item Select the statement whose translation is
\begin{quote}
``Every integer has at least one nonzero factor."
\end{quote}
or write NONE if none of (i)-(viii) work.

\item Select the statement whose translation is
\begin{quote}
``There is an integer of which
all nonzero integers are a factor."
\end{quote}
or write NONE if none of (i)-(viii) work.

\item For each  statement (i)-(viii), determine
if  it is true or  false.
\end{enumerate}     \item \hspace{1in}\\ 

Which of the following formalizes the definition of the predicate
$Pr(x)$ over the set of integers, and evaluates to $T$ exactly when 
$x$ is prime. (Select all and only correct options.)

\begin{enumerate}
    \item $\forall a \in \mathbb{Z}^{\neq 0}~( ~(x > 1 \land a >0) \to F(~(a,x)~))$
    \item $\lnot \exists a \in \mathbb{Z}^{\neq 0} ~(x > 1 \land (a=1 \lor a=x) \land F(~(a,x)))$
    \item $(x > 1) \land \forall a \in \mathbb{Z}^{\neq 0}~( ~(~ a>0 \land F(~(a,x)~)~) \to (a=1 \lor a=x)~)$
    \item $(x > 1) \land \forall a \in \mathbb{Z}^{\neq 0}~( ~(~ a>1 \land \lnot (a=x) ~) \to \lnot F(~(a,x)~)~)$
\end{enumerate} \end{enumerate}

\end{document}
