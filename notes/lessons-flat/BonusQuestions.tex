\documentclass[12pt, oneside]{article}

\usepackage[letterpaper, scale=0.89, centering]{geometry}
\usepackage{fancyhdr}
\setlength{\parindent}{0em}
\setlength{\parskip}{1em}

\pagestyle{fancy}
\fancyhf{}
\renewcommand{\headrulewidth}{0pt}
\rfoot{\href{https://creativecommons.org/licenses/by-nc-sa/2.0/}{CC BY-NC-SA 2.0} Version \today~(\thepage)}

\usepackage{amssymb,amsmath,pifont,amsfonts,comment,enumerate,enumitem}
\usepackage{currfile,xstring,hyperref,tabularx,graphicx,wasysym}
\usepackage[labelformat=empty]{caption}
\usepackage[dvipsnames,table]{xcolor}
\usepackage{multicol,multirow,array,listings,tabularx,lastpage,textcomp,booktabs}

\lstnewenvironment{algorithm}[1][] {   
    \lstset{ mathescape=true,
        frame=tB,
        numbers=left, 
        numberstyle=\tiny,
        basicstyle=\rmfamily\scriptsize, 
        keywordstyle=\color{black}\bfseries,
        keywords={,procedure, div, for, to, input, output, return, datatype, function, in, if, else, foreach, while, begin, end, }
        numbers=left,
        xleftmargin=.04\textwidth,
        #1
    }
}
{}
\lstnewenvironment{java}[1][]
{   
    \lstset{
        language=java,
        mathescape=true,
        frame=tB,
        numbers=left, 
        numberstyle=\tiny,
        basicstyle=\ttfamily\scriptsize, 
        keywordstyle=\color{black}\bfseries,
        keywords={, int, double, for, return, if, else, while, }
        numbers=left,
        xleftmargin=.04\textwidth,
        #1
    }
}
{}

\newcommand\abs[1]{\lvert~#1~\rvert}
\newcommand{\st}{\mid}

\newcommand{\A}[0]{\texttt{A}}
\newcommand{\C}[0]{\texttt{C}}
\newcommand{\G}[0]{\texttt{G}}
\newcommand{\U}[0]{\texttt{U}}

\newcommand{\cmark}{\ding{51}}
\newcommand{\xmark}{\ding{55}}

 
\begin{document}
\begin{flushright}
    \StrBefore{\currfilename}{.}
\end{flushright} \newcommand{\KeyIdea}[1]{\begin{quote}{\it Key idea:~{#1}}\end{quote}}

\begin{enumerate}
    \item True or False: A recursive definition of the set of integers $\mathbb{Z}$ is 
    \[
        \begin{array}{ll}
        \textrm{Basis Step: } & 20 \in \mathbb{Z} \\
        \textrm{Recursive Step: } & \textrm{If } x \in \mathbb{Z} \textrm{, then } x-1 \in \mathbb{Z}
        \end{array}
    \]

        \KeyIdea{To define a set, we need to specify which elements are in and which elements are out.
        A description that gives some, but not all, integers is not a definition of the set of integers.}

    \item When $B = \{ \A, \C, \U\, \G\}$ is the set of RNA bases, 
    consider the recursively defined set $X$ given by:
    \begin{itemize}
    \item[] Basis Step: $\A\A \in X$, $\C\C \in X$, $\U\U \in X$, $\G\G \in X$
    \item[] Recursive Step: If $x \in X$ and $b \in B$ then $bxb \in X$
    \end{itemize}
    where $bxb$ is the result of string concatenations.

    Give three example elements of $X$ and an English description of the set.

        \KeyIdea{To build new example elements we can start with an element from 
        the basis step and then apply a rule from the recursive step finitely many times.}

    \item Write in roster method the set given by the Cartesian product
    \[ 
        \{1,2\} \times \{a,b,c\}
    \]

        \KeyIdea{Cartesian product does not require the sets to have the same size as one another
        or to have the same types of elements as one another.}

    \item Consider the function $d_2: \textrm{The set of ordered pairs of ratings $4$-tuples} \to \mathbb{R}$
    given by 
    \[ 
        d_2 (~( (x_1, x_2, x_3, x_4), (y_1, y_2, y_3, y_4)~)~) = \sqrt{\sum_{i=1}^4 (x_i - y_i)^2}
    \]
    Let $z = (1,1,1,1)$ be a ratings $4$-tuple and consider the set
    \[
        \{ x \mid d_2 ( ~(x,z)~) = 1 \}
    \]
    Rewrite this set using the roster method.

        \KeyIdea{The input to this function is an ordered pair each of whose components is a ratings $4$-tuple.}

    \item Rewrite the set
    \[
        \{ rnalen(x) \mid x \in S \textrm{ and } rnalen(x) < 2 \}
    \]
    using the roster method.

    \KeyIdea{The set builder definition for this set can be read as ``The collection of all outputs
    of the function $rnalen$ when the input is taken from the set of RNA strands for which 
    $rnalen$ gives value less than $2$.'' Informally, we consider just RNA strands whose $rnalen$ value
    is less than $2$ and collect their $rnalen$ values into our set.}
\end{enumerate}
\end{document}