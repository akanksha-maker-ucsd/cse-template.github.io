\documentclass[12pt, oneside]{article}

\usepackage[letterpaper, scale=0.89, centering]{geometry}
\usepackage{fancyhdr}
\setlength{\parindent}{0em}
\setlength{\parskip}{1em}

\pagestyle{fancy}
\fancyhf{}
\renewcommand{\headrulewidth}{0pt}
\rfoot{\href{https://creativecommons.org/licenses/by-nc-sa/2.0/}{CC BY-NC-SA 2.0} Version \today~(\thepage)}

\usepackage{amssymb,amsmath,pifont,amsfonts,comment,enumerate,enumitem}
\usepackage{currfile,xstring,hyperref,tabularx,graphicx,wasysym}
\usepackage[labelformat=empty]{caption}
\usepackage[dvipsnames,table]{xcolor}
\usepackage{multicol,multirow,array,listings,tabularx,lastpage,textcomp,booktabs}

\lstnewenvironment{algorithm}[1][] {   
    \lstset{ mathescape=true,
        frame=tB,
        numbers=left, 
        numberstyle=\tiny,
        basicstyle=\rmfamily\scriptsize, 
        keywordstyle=\color{black}\bfseries,
        keywords={,procedure, div, for, to, input, output, return, datatype, function, in, if, else, foreach, while, begin, end, }
        numbers=left,
        xleftmargin=.04\textwidth,
        #1
    }
}
{}
\lstnewenvironment{java}[1][]
{   
    \lstset{
        language=java,
        mathescape=true,
        frame=tB,
        numbers=left, 
        numberstyle=\tiny,
        basicstyle=\ttfamily\scriptsize, 
        keywordstyle=\color{black}\bfseries,
        keywords={, int, double, for, return, if, else, while, }
        numbers=left,
        xleftmargin=.04\textwidth,
        #1
    }
}
{}

\newcommand\abs[1]{\lvert~#1~\rvert}
\newcommand{\st}{\mid}

\newcommand{\A}[0]{\texttt{A}}
\newcommand{\C}[0]{\texttt{C}}
\newcommand{\G}[0]{\texttt{G}}
\newcommand{\U}[0]{\texttt{U}}

\newcommand{\cmark}{\ding{51}}
\newcommand{\xmark}{\ding{55}}

 
\begin{document}
\begin{flushright}
    \StrBefore{\currfilename}{.}
\end{flushright} 
\section*{Monday November 22}



{\bf Definition}: When $A$ and $B$ are sets, we say any subset of $A \times B$ is a {\bf binary relation}. 
A relation $R$ can also be represented as

\begin{itemize}
\item A function $f_{TF} : A \times B \to \{T, F\}$
where, for $a \in A$ and $b \in B$, $f_{TF}(~(a,b)~) = 
\begin{cases} 
    T \qquad&\text{when } (a,b) \in R \\
    F \qquad&\text{when } (a,b) \notin R
\end{cases}$
\item A function $f_{\mathcal{P}} : A   \to \mathcal{P}(B)$ where, for $a \in A$, 
$f_{\mathcal{P}}( a ) = \{ b \in B ~|~ (a,b) \in R \}$
\end{itemize}

When $A$ is a set, we say any subset of $A \times A$ is a (binary) {\bf relation} on $A$.

 

For relation $R$ on a set $A$, we can represent this relation as a
{\bf graph}: a collection of nodes (vertices) and edges (arrows). The 
nodes of the graph are the elements of $A$ and 
there is an edge from $a$ to $b$ exactly when $(a,b) \in R$.

\vspace{100pt} 

{\it Example}: For $A = \mathcal{P}(\mathbb{R})$, we can define the relation $EQ_{\mathbb{R}}$ on $A$ as 
\[
\{ (X_1, X_2 ) \in\mathcal{P}(\mathbb{R})  \times \mathcal{P}(\mathbb{R}) ~|~ |X_1| = |X_2| \}
\]

\vspace{50pt}

{\it Example}: Let $R_{(\textbf{mod } n)}$ be the set of all pairs of integers $(a, b)$ such that $(a \textbf{ mod } n = b \textbf{ mod } n)$.
Then $a$ is {\bf congruent to} $b$ \textbf{mod} $n$ means $(a, b) \in R_{(\textbf{mod } n)}$. A common notation is to write this as $a \equiv b (\textbf{mod } n)$.


$R_{(\textbf{mod } n)}$ is a relation on the set $\underline{\hspace{25em}}$


Some example elements of $R_{(\textbf{mod } 4)}$ are: 

\vspace{50pt} \newpage


A relation $R$ on a set $A$ is called {\bf reflexive} 
means $(a, a) \in R$ for every element $a \in A$. 
 

{\it Informally}, every element is related to itself.

{\it Graphically}, there are self-loops (edge from a node back to itself) at 
every node. \vfill


A relation $R$ on a set $A$ is called {\bf symmetric} means 
$(b, a) \in R$ whenever $(a, b) \in R$, for all $a, b \in A$. 
 

{\it Informally}, order doesn't matter for this relation.

{\it Graphically}, every edge has a paired ``backwards'' edge so we might
as well drop the arrows and think of edges as undirected. \vfill


A relation $R$ on a set $A$ is called {\bf transitive} means 
whenever $(a, b) \in R$ and $(b, c) \in R$, then $(a, c) \in R$, for all $a, b, c \in A$.
 

{\it Informally}, chains of relations collapse.

{\it Graphically}, there's a shortcut between any endpoints of a chain of 
edges. \vfill


A relation $R$ on a set $A$ is called {\bf antisymmetric} means 
$\forall a \in A ~\forall b \in A~\left(~\left( ~(a,b) \in R \land (b,a) \in R ~\right) \to a=b~\right)$
 

{\it Informally}: the relation has directionality.

{\it Graphically}, can organize the nodes of the graph so that 
all non-self loop edges go up. \vfill
\newpage



When the domain is $\{ a,b,c,d,e,f,g,h\}$ define a relation that is {\bf not reflexive} and 
is {\bf not symmetric} and is {\bf not transitive}.

\vspace{80pt}

When the domain is $\{ a,b,c,d,e,f,g,h\}$ define a relation that is {\bf not reflexive} but 
is {\bf symmetric} and is {\bf transitive}.

\vspace{80pt}


When the domain is $\{ a,b,c,d,e,f,g,h\}$ define a relation that is {\bf symmetric} and
is {\bf antisymmetric}.

\vspace{80pt}

Is the relation $EQ_{\mathbb{R}}$ reflexive? symmetric? transitive? antisymmetric?

\vspace{80pt}

Is the relation $R_{(\textbf{mod } 4)}$ reflexive? symmetric? transitive? antisymmetric?

\vspace{80pt}

Is the relation $Sub$ on $W = \mathcal{P}(\{1,2,3,4,5\})$ given by $Sub = \{ (X,Y) \mid X \subseteq Y \}$
reflexive? symmetric? transitive? antisymmetric?

\vspace{80pt}
 \newpage


A relation is an {\bf equivalence relation} means it is reflexive, symmetric, and transitive. 

A relation is a {\bf partial ordering} (or partial order) means 
it is reflexive, antisymmetric, and transitive. 

For a partial ordering, its {\bf Hasse diagram} is a graph whose nodes (vertices) are the elements of the 
domain of the binary relation and which are located such that nodes connected to nodes
above them by (undirected) edges indicate that the relation holds between the lower node and the higher node. 
Moreover, the diagram omits self-loops and omits edges that are guaranteed by transitivity.
 

Draw the Hasse diagram of the partial order on the set $\{a,b,c,d,e,f,g\}$ defined as
\begin{align*}
    \{  &(a,a), (b,b), (c,c), (d,d), (e,e), (f,f), (g,g), \\
        &(a,c), (a,d), (d,g), (a,g), (b,f), (b,e), (e,g), (b,g) \}
\end{align*}

\vspace{100pt}

%
 \vfill

{\it Summary}: binary relations can be useful for organizing elements in a domain. 
Some binary relations have special properties that make them act like some familiar relations.
Equivalence relations (reflexive, symmetric, transitive binary relations) ``act like'' equals.
Partial orders (reflexive, antisymmetric, transitive binary relations) ``act like'' less than or equals to.
\newpage
\section*{Review}
\begin{enumerate}
    \item \hspace{1in}\\ 

Recall that the binary relation $EQ_{\mathbb{R}}$ on $\mathcal{P}(\mathbb{R})$ is
\[
\{ (X_1, X_2 ) \in\mathcal{P}(\mathbb{R})  \times \mathcal{P}(\mathbb{R}) ~|~ |X_1| = |X_2| \}
\]
and $R_{(\textbf{mod } n)}$ is the set of all pairs of integers $(a, b)$ 
such that $(a \textbf{ mod } n = b \textbf{ mod } n)$.

Select all and only the correct items.
\begin{enumerate}
\item $(\mathbb{Z}, \mathbb{R}) \in EQ_{\mathbb{R}}$
\item $(0,1) \in EQ_{\mathbb{R}}$
\item $(\emptyset, \emptyset) \in EQ_{\mathbb{R}}$
\item $(-1,1) \in R_{(\textbf{mod } 2)}$
\item $(1,-1) \in R_{(\textbf{mod } 3)}$ 
\item $(4, 16, 0) \in R_{\textbf{(mod } 4)}$ 
\end{enumerate}     \item \hspace{1in}\\ 

Consider the binary relation on $\mathbb{Z}^+$ defined by $\{(a,b) ~|~ \exists c \in \mathbb{Z} ( b = ac)\}$.
Select all and only the properties that this binary relation has.
\begin{enumerate}
\item It is reflexive.
\item It is symmetric.
\item It is transitive.
\item It is antisymmetric.
\end{enumerate}     \item \hspace{1in}\\ 

\begin{enumerate}
\item Consider the partial order on the set $\mathcal{P}(\{1,2,3\})$ given by the binary relation 
    $\{ (X,Y) ~|~X \subseteq Y \}$
    \begin{enumerate}
    \item How many nodes are in the Hasse diagram of this partial order?
    \item How many edges are in the Hasse diagram of this partial order?
    \end{enumerate}
\item Consider the binary relation on $\{1,2,4,5,10,20\}$ 
defined by $\{(a,b) ~|~ \exists c \in \mathbb{Z} ( b = ac)\}$.
    \begin{enumerate}
    \item How many nodes are in the Hasse diagram of this partial order?
    \item How many edges are in the Hasse diagram of this partial order?
    \end{enumerate}
\end{enumerate} \end{enumerate}
\newpage

\section*{Wednesday November 24}
\subsection*{Exploring equivalence relations}


A {\bf partition} of a set $A$ is a set of non-empty, disjoint subsets 
$A_1, A_2, \cdots, A_n$ such that 
\[
    A = \bigcup_{i=1}^{n} A_i = \{ x \mid \exists i (x \in A_i) \}
\] 

An {\bf equivalence class} of an element $a \in A$ 
with respect to an equivalence relation $R$ on the set $A$ is the set 
\[
    \{s \in A | (a, s) \in R \}
\] 
We write $[a]_R$ for this set, which is the equivalence class of $a$ with respect to $R$. 
{\bf Fact}: When $R$ is an equivalence relation on a nonempty set $A$, 
the collection of equivalence classes of $R$ is a partition of $A$.

Also, given a partition $P$ of $A$, the relation $R_P$ on $A$ given by 
\[
    \{ (x,y) \in A \times A ~|~ \text{$x$ and $y$ are in the same part of the partition $R_P$}\}
\]
is an equivalence relation on $A$. 

{\it Recall}: We say $a$ is {\bf congruent to} $b$ \textbf{mod} $n$ 
means $(a, b) \in R_{(\textbf{mod } n)}$. 
A common notation is to write this as $a \equiv b (\textbf{mod } n)$.

We can partition the set of integers using equivalence classes of  $R_{(\textbf{mod } 4)}$

\begin{align*}
    [0]_{R_{(\textbf{mod } 4)}} &= \phantom{ \{ x \in \mathbb{Z} \mid x \equiv 0 ((\textbf{mod } 4)) \} 
    = \{ x \in \mathbb{Z} \mid x \textbf{ mod } 4 = 0 \textbf{ mod } 4 = 0 \} = \{ 4c \mid c \in \mathbb{Z}\} }\\
    [1]_{R_{(\textbf{mod } 4)}} &= \phantom{ \{ x \in \mathbb{Z} \mid x \equiv 1 ((\textbf{mod } 4)) \} 
    = \{ x \in \mathbb{Z} \mid x \textbf{ mod } 4 = 1 \textbf{ mod } 4 = 1 \} = \{ 4c+1 \mid c \in \mathbb{Z}\} }\\
    [2]_{R_{(\textbf{mod } 4)}} &= \phantom{ \{ x \in \mathbb{Z} \mid x \equiv 2 ((\textbf{mod } 4)) \} 
    = \{ x \in \mathbb{Z} \mid x \textbf{ mod } 4 = 2 \textbf{ mod } 4 = 0 \} = \{ 4c+2 \mid c \in \mathbb{Z}\} }\\
    [3]_{R_{(\textbf{mod } 4)}} &= \phantom{ \{ x \in \mathbb{Z} \mid x \equiv 3 ((\textbf{mod } 4)) \} 
    = \{ x \in \mathbb{Z} \mid x \textbf{ mod } 4 = 3 \textbf{ mod } 4 = 3 \} = \{ 4c+3 \mid c \in \mathbb{Z}\} }\\
    [4]_{R_{(\textbf{mod } 4)}} &= \phantom{ \{ x \in \mathbb{Z} \mid x \equiv 4 ((\textbf{mod } 4)) \} 
    = \{ x \in \mathbb{Z} \mid x \textbf{ mod } 4 = 4 \textbf{ mod } 4 = 0 \} = \{ 4c \mid c \in \mathbb{Z}\} }\\
    [5]_{R_{(\textbf{mod } 4)}} &= \phantom{ \{ x \in \mathbb{Z} \mid x \equiv 5 ((\textbf{mod } 4)) \} 
    = \{ x \in \mathbb{Z} \mid x \textbf{ mod } 4 = 5 \textbf{ mod } 4 = 1 \} = \{ 4c+1 \mid c \in \mathbb{Z}\} }\\
    [-1]_{R_{(\textbf{mod } 4)}} &= \phantom{ \{ x \in \mathbb{Z} \mid x \equiv -1 ((\textbf{mod } 4)) \} 
    = \{ x \in \mathbb{Z} \mid x \textbf{ mod } 4 = -1 \textbf{ mod } 4 = 3 \} = \{ 4c+3 \mid c \in \mathbb{Z}\} }
\end{align*}
\[
\mathbb{Z} =  [0]_{R_{(\textbf{mod } 4)}}~ \cup ~[1]_{R_{(\textbf{mod } 4)}} ~\cup~[2]_{R_{(\textbf{mod } 4)}}~\cup~
[3]_{R_{(\textbf{mod } 4)}}
\]





 

Integers are useful because they can be used to encode other objects
and have multiple representations. However, infinite sets are sometimes
expensive to work with computationally. Reducing our attention to 
a {\it partition of the integers} based on congrunce mod $n$, where
each part is represented by a (not too large) integer gives a useful 
compromise where many algebraic properties of the integers are preserved, 
and we also get the benefits of a finite domain. Moreover, modular arithmetic
is well-suited to model any cyclic behavior. 

{\bf Lemma} : For $a, b \in \mathbb{Z}$ 
and positive integer $n$, $(a,b) \in R_{(\textbf{mod } n)}$ if and only if  $n | a-b$.

{\bf Proof}: 

\phantom{Consider arbitrary integers $a,b$ and arbitrary positive integer $n$.}

\phantom{Assume $a \textbf{ mod } n = b \textbf{ mod } n$. Call this 
remainder $r$ and we have integers $q_1, q_2$ such that $a = q_1 n + r$
and $b = q_2 n + r$. Calculating $a-b = (q_1 n + r) - (q_2n +r) = (q_1 - q_2)n$ 
an integer multiple of $n$, as required.}

\phantom{Assume there is integer $c$ with $a-b = cn$. By long division 
there are integers $q$ and $r$ ($0 \leq r < n$) with $b = qn + r$. Then 
$a = b + cn = qn+r + cn = (q+c)n + r$. Since long division gives a unique remainder,
this means $a \textbf{ mod } n = r = b \textbf{ mod } n$, as required.}

\vspace{200pt} 

{\bf Application: Cycling}

How many minutes past the hour are we at?  \hfill {\it Model with} $+15 \textbf{ mod } 60$

\begin{tabular}{lccccccccccc}
{\bf Time:} &12:00pm  &12:15pm&12:30pm  &12:45pm&1:00pm  &1:15pm&1:30pm  &1:45pm&2:00pm \\
{\bf ``Minutes past":} &$0$ & $15$ & $30$ & $45$ &$0$ & $15$ & $30$ & $45$ &$0$\\
\end{tabular}

\vspace{50pt}

Replace each English letter by a letter that's fifteen ahead of it in the alphabet
  \hfill {\it Model with} $+15 \textbf{ mod } 26$

{\tiny
\begin{tabular}{lcccccccccccccccccccccccccc}
{\bf Original index:} & $0$ & $1$
 & $2$ & $3$ &  $4$ & $5$ &  $6$ & $7$ &  $8$ & $9$ & $10$ & $11$ & $12$ & $13$ & $14$ & $15$ & 
  $16$ & $17$ &  $18$ & $19$ &  $20$ & $21$ &  $22$ & $23$ & $24$ & $25$\\
{\bf Original letter:} & A & B& C & D & E & F& G& H & I & J & K & L &M & N& O &P &Q & R & S & T & U & V & W & X & Y & Z \\
{\bf Shifted letter}: &P &Q & R & S & T & U & V & W & X & Y & Z & A & B& C & D & E & F& G& H & I & J & K & L &M & N& O \\
{\bf Shifted index:} &$15$ & 
  $16$ & $17$ &  $18$ & $19$ &  $20$ & $21$ &  $22$ & $23$ & $24$ & $25$ & $0$ & $1$
 & $2$ & $3$ &  $4$ & $5$ &  $6$ & $7$ &  $8$ & $9$ & $10$ & $11$ & $12$ & $13$ & $14$ 
\end{tabular}
} 

{\bf Modular arithmetic}: 

{\bf Lemma}: For $a, b \in \mathbb{Z}$ 
and positive integer $n$, if $a \equiv b ~(\textbf{ mod } n)$ and $c \equiv d ~(\textbf{ mod } n)$ 
then $a+c \equiv b+d ~(\textbf{ mod } n)$ and $ac \equiv bd ~(\textbf{ mod } n)$.
{\bf Informally}: can bring mod ``inside" and do it first, for addition and for multiplication.


$(102 + 48) \textbf{ mod } 10 = \underline{\phantom{\hspace{3in}}} $ 

$(7 \cdot 10) \textbf{ mod } 5 = \underline{\phantom{\hspace{3.3in}}} $ 

$(2^5) \textbf{ mod } 3 =  \underline{\phantom{\hspace{3.45in}}} $ 

\vfill

 \newpage
{\bf Application: Cryptography}

{\bf Definition}: Let $a$ be a positive integer and $p$ be a 
large\footnote{We leave the definition of ``large'' vague here, but 
think hundreds of digits for practical applications. In practice, 
we also need a particular relationship between $a$ and $p$ to hold, 
which we leave out here.} prime number, both known to everyone. 
Let $k_1$ be a secret large number known only to person $P_1$ (Alice) 
and $k_2$ be a secret large number known only to person $P_2$ (Bob). 
Let the {\bf Diffie-Helman shared key} for $a, p, k_1, k_2$ be 
$(a^{k_1\cdot k_2} \textbf{ mod } p)$.


{\bf Idea}: $P_1$ can quickly compute the Diffie-Helman shared key 
knowing only $a, p, k_1$ and the result of $a^{k_2} \textbf{ mod } p$ 
(that is, $P_1$ can compute the shared key without knowing $k_2$, 
only $a^{k_2} \textbf{ mod } p$). Similarly, $P_2$ can 
quickly compute the Diffie-Helman shared key knowing only 
$a, p, k_2$ and the result of $a^{k_1} \textbf{ mod } p$ 
(that is, $P_2$ can compute the shared key without knowing $k_1$, 
only $a^{k_1} \textbf{ mod } p$). But, any person $P_3$ who 
knows neither $k_1$ nor $k_2$ (but may know any and all of the other values) 
cannot compute the shared secret efficiently.

{\bf Key property for *shared* secret}: 
\[
    \forall a \in \mathbb{Z} \, \forall b \in \mathbb{Z} \, \forall g \in \mathbb{Z}^+ \, 
    \forall n \in \mathbb{Z}^+ ((g^a \textbf{ mod } n)^b, (g^b \textbf{ mod } n)^a) \in R_{(\textbf{mod } n)}
\]

{\bf Key property for shared *secret*}:

There are efficient algorithms to calculate the result of modular exponentiation 
but there are no (known) efficient algorithms to calculate discrete logarithm. \newpage
\subsection*{Review}
\begin{enumerate}
    \item \hspace{1in}\\ 

Fill in the blanks in the following proof that, for any equivalence relation $R$ on a set $A$,
\[
\forall a \in A ~\forall b \in A~\left( (a,b) \in R \leftrightarrow [a]_R\cap [b]_R \neq \emptyset \right)
\]

{\bf Proof}: Towards a  \textbf{(a)}$\underline{\phantom{\hspace{1.3in}}}$, consider arbitrary elements $a$, $b$ in $A$. We will 
prove the biconditional statement by proving each direction of the conditional in turn.

{\bf Goal 1}: we need to show $(a,b) \in R \to [a]_R\cap [b]_R \neq \emptyset$
{\it Proof of Goal 1}: Assume towards a \textbf{(b)}$\underline{\phantom{\hspace{1.3in}}}$ 
that $(a,b) \in R$. We will work to show
that $[a]_R\cap [b]_R \neq \emptyset$. Namely, we need an element that is in both equivalence classes, that is, we
 need to prove the existential claim $\exists x \in A ~(x \in [a]_{R} \land x \in [b]_{R})$. 
 Towards a \textbf{(c)}$\underline{\phantom{\hspace{1.3in}}}$, consider $x = b$, 
 an element of $A$ by definition. By \textbf{(d)}$\underline{\phantom{\hspace{1.3in}}}$  of $R$, we know that $(b,b) \in R$ 
 and thus, $b \in [b]_{R}$.
 By assumption in this proof, we have that $(a,b) \in R$, and so by  definition of equivalence classes, $b \in [a]_R$.
 Thus, we have proved both conjuncts and this part of the proof is complete.
 
{\bf Goal 2}: we need to show $[a]_R\cap [b]_R \neq \emptyset \to (a,b) \in R $
{\it Proof of Goal 2}: Assume towards a \textbf{(e)}$\underline{\phantom{\hspace{1.3in}}}$ 
that $[a]_R\cap [b]_R \neq \emptyset $. We will work to show
that $(a,b) \in R$. By our assumption, the existential claim $\exists x \in A ~(x \in [a]_{R} \land x \in [b]_{R})$
is true. Call $w$ a witness; thus, $w \in [a]_R$ and $w \in [b]_R$. 
By  definition of equivalence classes, $w \in [a]_R$ means $(a,w) \in R$ and $w \in [b]_R$ means $(b,w) \in R$.
By \textbf{(f)}$\underline{\phantom{\hspace{1.3in}}}$  of $R$, $(w,b) \in R$. By 
\textbf{(g)}$\underline{\phantom{\hspace{1.3in}}}$ of $R$, since $(a,w) \in R$ and $(w,b) \in R$, we have that
$(a,b) \in R$, as required for  this part of the proof.
 
Consider the following expressions as options to fill in the two proofs above. Give your answer as one of the numbers below for each blank a-c. You may use some numbers for more than one blank, but each letter only uses one of the expressions below.

\begin{multicols}{2}
\begin{enumerate}[label=\roman*]
\item exhaustive proof
\item proof by universal generalization
\item proof of existential using a witness
\item proof by cases
\item direct proof
\item proof by contrapositive
\item proof by contradiction
\item reflexivity
\item symmetry
\item transitivity
\end{enumerate}
\end{multicols}     \item \hspace{1in}\\ 

Modular exponentiation is required to carry out the Diffie-Helman protocol for 
computing a shared secret over an unsecure channel.

Consider the following algorithm for fast exponentiation (based on binary 
expansion of the exponent).

    \begin{algorithm}[caption={Modular Exponentation}]
    procedure $modular~exponentiation$($b$: integer; 
                 $n = (a_{k-1}a_{k-2} \ldots a_1 a_0)_2$, $m$: positive integers)
    $x$ := $1$
    $power$ := $b$ mod $m$
    for $i$:= $0$ to $k-1$
      if $a_i = 1$ then $x$:= $(x \cdot power)$ mod $m$
      $power$ := $(power \cdot power)$ mod $m$
    return $x$ $\{x~\textrm{equals}~b^n \textbf{ mod } m\} $
    \end{algorithm}
    
    \begin{enumerate}
        \item If we wanted to calculate $3^8 \textbf{ mod } 7$ 
        using the modular exponentation algorithm above, what are the values of 
        the parameters $b$, $n$, and $m$?  (Write these values in usual, 
        decimal-like, mathematical notation.)
        \item Give the output of the $modular~exponentiation$ algorithm 
        with these parameters, i.e.\ calculate $3^8 \textbf{ mod } 7$.
        (Write these values in usual, 
        decimal-like, mathematical notation.)
    \end{enumerate} \end{enumerate}

\newpage
\section*{Friday November 26}

No class, in observance of Thanksgiving holiday.
\end{document}