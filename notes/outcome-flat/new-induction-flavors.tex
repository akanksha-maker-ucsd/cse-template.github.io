\documentclass[12pt, oneside]{article}

\usepackage[letterpaper, scale=0.89, centering]{geometry}
\usepackage{fancyhdr}
\setlength{\parindent}{0em}
\setlength{\parskip}{1em}

\pagestyle{fancy}
\fancyhf{}
\renewcommand{\headrulewidth}{0pt}
\rfoot{\href{https://creativecommons.org/licenses/by-nc-sa/2.0/}{CC BY-NC-SA 2.0} Version \today~(\thepage)}

\usepackage{amssymb,amsmath,pifont,amsfonts,comment,enumerate,enumitem}
\usepackage{currfile,xstring,hyperref,tabularx,graphicx,wasysym}
\usepackage[labelformat=empty]{caption}
\usepackage[dvipsnames,table]{xcolor}
\usepackage{multicol,multirow,array,listings,tabularx,lastpage,textcomp,booktabs}

\lstnewenvironment{algorithm}[1][] {   
    \lstset{ mathescape=true,
        frame=tB,
        numbers=left, 
        numberstyle=\tiny,
        basicstyle=\rmfamily\scriptsize, 
        keywordstyle=\color{black}\bfseries,
        keywords={,procedure, div, for, to, input, output, return, datatype, function, in, if, else, foreach, while, begin, end, }
        numbers=left,
        xleftmargin=.04\textwidth,
        #1
    }
}
{}
\lstnewenvironment{java}[1][]
{   
    \lstset{
        language=java,
        mathescape=true,
        frame=tB,
        numbers=left, 
        numberstyle=\tiny,
        basicstyle=\ttfamily\scriptsize, 
        keywordstyle=\color{black}\bfseries,
        keywords={, int, double, for, return, if, else, while, }
        numbers=left,
        xleftmargin=.04\textwidth,
        #1
    }
}
{}

\newcommand\abs[1]{\lvert~#1~\rvert}
\newcommand{\st}{\mid}

\newcommand{\A}[0]{\texttt{A}}
\newcommand{\C}[0]{\texttt{C}}
\newcommand{\G}[0]{\texttt{G}}
\newcommand{\U}[0]{\texttt{U}}

\newcommand{\cmark}{\ding{51}}
\newcommand{\xmark}{\ding{55}}

 
\begin{document}
\begin{flushright}
    \StrBefore{\currfilename}{.}
\end{flushright} \section*{Fundamental theorem proof}


{\bf Theorem}: Every positive integer {\it greater than 1} is a product of (one or more) primes.

{\bf Before we prove, let's try some examples}:

$20 = $

$100 = $

$5 = $


{\bf Proof by strong induction}, with $b=2$ and $j=0$.

{\bf Basis step}:  WTS property is true about  $2$.

Since $2$ is itself prime,
it is already written as a product of (one) prime.


{\bf Recursive step}: Consider an arbitrary integer $n \geq 2$.
Assume (as the strong induction hypothesis, IH) that the property is true about  each of $2, \ldots, n$.  
WTS that the property is true about  $n+1$: We want to show that $n+1$ can be written 
as a product of primes.  Notice that $n+1$ is itself prime or it is composite.

{\it Case 1}: assume $n+1$ is prime and then immediately it is written as a product
of (one) prime so we are done.  

{\it Case 2}: assume that $n+1$ is composite
so there are integers $x$ and $y$ where $n+1 = xy$ and each of them is between $2$ and $n$
(inclusive).  Therefore, the induction hypothesis applies to each of $x$ and $y$ so each 
of these factors of $n+1$ can be written as a product of primes.  Multiplying these products together, 
we get a product of primes that gives $n+1$, as required. 

Since both cases give the necessary
conclusion, the proof by cases for the recursive step is complete. \vfill
\section*{Fundamental theorem proof}


{\bf Theorem}: Every positive integer {\it greater than 1} is a product of (one or more) primes.

{\bf Before we prove, let's try some examples}:

$20 = $

$100 = $

$5 = $


{\bf Proof by strong induction}, with $b=2$ and $j=0$.

{\bf Basis step}:  WTS property is true about  $2$.

Since $2$ is itself prime,
it is already written as a product of (one) prime.


{\bf Recursive step}: Consider an arbitrary integer $n \geq 2$.
Assume (as the strong induction hypothesis, IH) that the property is true about  each of $2, \ldots, n$.  
WTS that the property is true about  $n+1$: We want to show that $n+1$ can be written 
as a product of primes.  Notice that $n+1$ is itself prime or it is composite.

{\it Case 1}: assume $n+1$ is prime and then immediately it is written as a product
of (one) prime so we are done.  

{\it Case 2}: assume that $n+1$ is composite
so there are integers $x$ and $y$ where $n+1 = xy$ and each of them is between $2$ and $n$
(inclusive).  Therefore, the induction hypothesis applies to each of $x$ and $y$ so each 
of these factors of $n+1$ can be written as a product of primes.  Multiplying these products together, 
we get a product of primes that gives $n+1$, as required. 

Since both cases give the necessary
conclusion, the proof by cases for the recursive step is complete. \vfill
\end{document}