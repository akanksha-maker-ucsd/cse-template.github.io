\documentclass[12pt, oneside]{article}

\usepackage[letterpaper, scale=0.89, centering]{geometry}
\usepackage{fancyhdr}
\setlength{\parindent}{0em}
\setlength{\parskip}{1em}

\pagestyle{fancy}
\fancyhf{}
\renewcommand{\headrulewidth}{0pt}
\rfoot{\href{https://creativecommons.org/licenses/by-nc-sa/2.0/}{CC BY-NC-SA 2.0} Version \today~(\thepage)}

\usepackage{amssymb,amsmath,pifont,amsfonts,comment,enumerate,enumitem}
\usepackage{currfile,xstring,hyperref,tabularx,graphicx,wasysym}
\usepackage[labelformat=empty]{caption}
\usepackage[dvipsnames,table]{xcolor}
\usepackage{multicol,multirow,array,listings,tabularx,lastpage,textcomp,booktabs}

\lstnewenvironment{algorithm}[1][] {   
    \lstset{ mathescape=true,
        frame=tB,
        numbers=left, 
        numberstyle=\tiny,
        basicstyle=\rmfamily\scriptsize, 
        keywordstyle=\color{black}\bfseries,
        keywords={,procedure, div, for, to, input, output, return, datatype, function, in, if, else, foreach, while, begin, end, }
        numbers=left,
        xleftmargin=.04\textwidth,
        #1
    }
}
{}
\lstnewenvironment{java}[1][]
{   
    \lstset{
        language=java,
        mathescape=true,
        frame=tB,
        numbers=left, 
        numberstyle=\tiny,
        basicstyle=\ttfamily\scriptsize, 
        keywordstyle=\color{black}\bfseries,
        keywords={, int, double, for, return, if, else, while, }
        numbers=left,
        xleftmargin=.04\textwidth,
        #1
    }
}
{}

\newcommand\abs[1]{\lvert~#1~\rvert}
\newcommand{\st}{\mid}

\newcommand{\A}[0]{\texttt{A}}
\newcommand{\C}[0]{\texttt{C}}
\newcommand{\G}[0]{\texttt{G}}
\newcommand{\U}[0]{\texttt{U}}

\newcommand{\cmark}{\ding{51}}
\newcommand{\xmark}{\ding{55}}

 
\begin{document}
\begin{flushright}
    \StrBefore{\currfilename}{.}
\end{flushright} \section*{Definitions}


\begin{center}
\begin{tabular}{|llp{9.8cm}|}
\hline
{\bf Term} & {\bf Notation Example(s)} & {\bf We say in English \ldots } \\
\hline
sequence & $x_1, \ldots, x_n$ & A sequence $x_1$ to $x_n$ \\
summation & $\sum_{i=1}^n x_i$ or $\displaystyle{\sum_{i=1}^n x_i}$ & The sum of the terms of the sequence $x_1$ to $x_n$ \\
&&\\
all reals & $\mathbb{R}$ & The (set of all) real numbers (numbers on the number line)\\
all integers & $\mathbb{Z}$ & The (set of all) integers (whole numbers including negatives, zero, and positives) \\
all positive integers & $\mathbb{Z}^+$ & The (set of all) strictly positive integers \\
all natural numbers & $\mathbb{N}$ & The (set of all) natural numbers. {\bf Note}: we use the convention that $0$ is a natural number. \\
&&\\
piecewise rule definition & $f(x) = \begin{cases} x & \text{if~}x \geq 0 \\ -x & \text{if~}x<0\end{cases}$ &
Define $f$ of $x$ to be $x$ when $x$ is nonnegative and to be $-x$ when $x$ is negative\\
function application & $f(7)$ & $f$ of $7$ {\bf or} $f$ applied to $7$ {\bf or} the image of $7$ under $f$\\
                     & $f(z)$ & $f$ of $z$ {\bf or} $f$ applied to $z$ {\bf or} the image of $z$ under $f$\\
                     & $f(g(z))$ & $f$ of $g$ of $z$ {\bf or} $f$ applied to the result of $g$ applied to $z$ \\
&&\\
absolute value & $\lvert -3 \rvert$ & The absolute value of $-3$ \\
square root & $\sqrt{9}$ & The non-negative square root of $9$ \\
&&\\


\hline
\end{tabular}
\end{center} \vfill
\section*{Data types}


\begin{center}
    \begin{tabular}{p{4.6in}p{2.6in}}
    {\bf  Term} & {\bf Examples}:\\
    &  (add additional examples from class)\\
    \hline 
    {\bf set} \newline
    unordered collection of elements & $7 \in \{43, 7, 9 \}$ \qquad $2 \notin \{43, 7, 9 \}$ \\
    {\it repetition doesn't matter} & \\
    {\it Equal sets agree on membership of all elements}& \\
    \hline
    {\bf $n$-tuple} \newline
    ordered sequence of elements with $n$ ``slots" ($n >0$) & \\
    {\it repetition matters, fixed length} &\\
    {\it Equal $n$-tuples have corresponding components equal}& \\
    \hline
    {\bf string} \newline
    ordered finite sequence of elements each from specified
    set & \\
    {\it repetition matters, arbitrary finite length} &\\
    {\it Equal strings have same length and corresponding characters equal}
    \end{tabular}
\end{center}

{\it Special cases}: 

When $n=2$, the 2-tuple is called an {\bf ordered pair}.

A string of length $0$ is called the {\bf empty string} and is denoted $\lambda$.

A set with no elements is called the {\bf empty set} and is denoted $\{\}$ or $\emptyset$. \vfill
\section*{Defining sets}


{\it To define sets:}

To define a set using {\bf roster method}, explicitly list its elements. That is,
start with $\{$ then list elements of 
the set separated by commas and close with $\}$.

To define a set using {\bf set builder definition}, either form 
``The set of all $x$ from the universe $U$ such that $x$ is ..." by writing
\[\{x \in U \mid ...x... \}\]
or form ``the collection of all outputs of some operation when the input ranges over the universe $U$"
by writing
\[\{ ...x... \mid x\in U \}\]

We use the symbol $\in$ as ``is an element of'' to indicate membership in a set.\\


{\bf Example sets}: For each of the following, identify whether it's defined using the roster method
or set builder notation and give an example element.
\begin{itemize}
    \item[]$\{ -1, 1\}$\\
    \item[]$\{0, 0 \}$\\
    \item[]$\{-1, 0, 1 \}$\\
    \item[]$\{(x,x,x) \mid x \in \{-1,0,1\} \}$\\
    \item[]$\{ \}$\\
    \item[]$\{ x \in \mathbb{Z} \mid x \geq 0 \}$\\
    \item[]$\{ x \in \mathbb{Z}  \mid x > 0 \}$\\
    \item[]$\{\A,\C,\U,\G\}$ \\
    \item[]$\{\A\U\G, \U\A\G, \U\G\A, \U\A\A \}$\\
\end{itemize}
 \vfill
\section*{Set operations}


\fbox{\parbox{\textwidth}{To define a set we can use the roster method, set builder notation, a recursive definition, 
and also we can apply a set operation to other sets. \\

{\bf New! Cartesian product of sets} and {\bf set-wise concatenation of sets of strings}\\


{\bf Definition}: Let $X$ and $Y$ be sets.  The {\bf Cartesian product} of $X$ and $Y$, denoted
$X \times Y$, is the set of all ordered pairs $(x,y)$ where $x \in X$ and $y \in Y$
\[
X \times Y = \{ (x,y) \mid x \in X \text{ and } y \in Y \}
\]
{\bf Definition}: Let $X$ and $Y$ be sets of strings over the same alphabet. The {\bf set-wise concatenation} 
of $X$ and $Y$, denoted $X \circ Y$, is the set of all results of string concatenation $xy$ where $x \in X$ 
and $y \in Y$
\[
X \circ Y = \{ xy \mid x \in X \text{ and } y \in Y \}
\]
}}

{\bf Pro-tip}: the meaning of writing one element next to another like $xy$ depends on the data-types of $x$ and 
$y$. When $x$ and $y$ are strings, the convention is that $xy$ is the result of string concatenation. 
When $x$ and $y$ are numbers, the convention is that $xy$ is the result of multiplication. This is 
(one of the many reasons) why is it very important to declare the data-type of variables before we use them.

{\it Fill in the missing entries in the table}:

\begin{center}
\begin{tabular}{cc}
{\bf  Set} & {\bf Example elements in this set}:\\
\hline 
& \\
$B$ &\A \qquad \C \qquad \G \qquad \U \\
& \\
\hline
& \\
\phantom{$B \times B$} & $(\A, \C)$ \qquad $(\U, \U)$\\
& \\
\hline
& \\
$B \times \{-1,0,1\}$ & \\
& \\
\hline
& \\
$\{-1,0,1\} \times B$ & \\
& \\
\hline
& \\
\phantom{$\{-1,0,1\} \times \{-1,0,1\}  \times \{-1,0,1\} $} & \qquad $(0,0,0)$ \\
& \\
\hline
& \\
$ \{\A, \C, \G, \U \} \circ  \{\A, \C, \G, \U \}$& \\
& \\
\hline
& \\
\phantom{$\{G\} \circ \{G\} \circ \{G\}$} & \qquad $\G\G\G\G$ \\
& \\
\hline

\end{tabular}
\end{center}

\vfill \vfill
\section*{Definitions}


\begin{center}
\begin{tabular}{|llp{9.8cm}|}
\hline
{\bf Term} & {\bf Notation Example(s)} & {\bf We say in English \ldots } \\
\hline
sequence & $x_1, \ldots, x_n$ & A sequence $x_1$ to $x_n$ \\
summation & $\sum_{i=1}^n x_i$ or $\displaystyle{\sum_{i=1}^n x_i}$ & The sum of the terms of the sequence $x_1$ to $x_n$ \\
&&\\
all reals & $\mathbb{R}$ & The (set of all) real numbers (numbers on the number line)\\
all integers & $\mathbb{Z}$ & The (set of all) integers (whole numbers including negatives, zero, and positives) \\
all positive integers & $\mathbb{Z}^+$ & The (set of all) strictly positive integers \\
all natural numbers & $\mathbb{N}$ & The (set of all) natural numbers. {\bf Note}: we use the convention that $0$ is a natural number. \\
&&\\
piecewise rule definition & $f(x) = \begin{cases} x & \text{if~}x \geq 0 \\ -x & \text{if~}x<0\end{cases}$ &
Define $f$ of $x$ to be $x$ when $x$ is nonnegative and to be $-x$ when $x$ is negative\\
function application & $f(7)$ & $f$ of $7$ {\bf or} $f$ applied to $7$ {\bf or} the image of $7$ under $f$\\
                     & $f(z)$ & $f$ of $z$ {\bf or} $f$ applied to $z$ {\bf or} the image of $z$ under $f$\\
                     & $f(g(z))$ & $f$ of $g$ of $z$ {\bf or} $f$ applied to the result of $g$ applied to $z$ \\
&&\\
absolute value & $\lvert -3 \rvert$ & The absolute value of $-3$ \\
square root & $\sqrt{9}$ & The non-negative square root of $9$ \\
&&\\


\hline
\end{tabular}
\end{center} \vfill
\section*{Data types}


\begin{center}
    \begin{tabular}{p{4.6in}p{2.6in}}
    {\bf  Term} & {\bf Examples}:\\
    &  (add additional examples from class)\\
    \hline 
    {\bf set} \newline
    unordered collection of elements & $7 \in \{43, 7, 9 \}$ \qquad $2 \notin \{43, 7, 9 \}$ \\
    {\it repetition doesn't matter} & \\
    {\it Equal sets agree on membership of all elements}& \\
    \hline
    {\bf $n$-tuple} \newline
    ordered sequence of elements with $n$ ``slots" ($n >0$) & \\
    {\it repetition matters, fixed length} &\\
    {\it Equal $n$-tuples have corresponding components equal}& \\
    \hline
    {\bf string} \newline
    ordered finite sequence of elements each from specified
    set & \\
    {\it repetition matters, arbitrary finite length} &\\
    {\it Equal strings have same length and corresponding characters equal}
    \end{tabular}
\end{center}

{\it Special cases}: 

When $n=2$, the 2-tuple is called an {\bf ordered pair}.

A string of length $0$ is called the {\bf empty string} and is denoted $\lambda$.

A set with no elements is called the {\bf empty set} and is denoted $\{\}$ or $\emptyset$. \vfill
\section*{Defining sets}


{\it To define sets:}

To define a set using {\bf roster method}, explicitly list its elements. That is,
start with $\{$ then list elements of 
the set separated by commas and close with $\}$.

To define a set using {\bf set builder definition}, either form 
``The set of all $x$ from the universe $U$ such that $x$ is ..." by writing
\[\{x \in U \mid ...x... \}\]
or form ``the collection of all outputs of some operation when the input ranges over the universe $U$"
by writing
\[\{ ...x... \mid x\in U \}\]

We use the symbol $\in$ as ``is an element of'' to indicate membership in a set.\\


{\bf Example sets}: For each of the following, identify whether it's defined using the roster method
or set builder notation and give an example element.
\begin{itemize}
    \item[]$\{ -1, 1\}$\\
    \item[]$\{0, 0 \}$\\
    \item[]$\{-1, 0, 1 \}$\\
    \item[]$\{(x,x,x) \mid x \in \{-1,0,1\} \}$\\
    \item[]$\{ \}$\\
    \item[]$\{ x \in \mathbb{Z} \mid x \geq 0 \}$\\
    \item[]$\{ x \in \mathbb{Z}  \mid x > 0 \}$\\
    \item[]$\{\A,\C,\U,\G\}$ \\
    \item[]$\{\A\U\G, \U\A\G, \U\G\A, \U\A\A \}$\\
\end{itemize}
 \vfill
\section*{Set operations}


\fbox{\parbox{\textwidth}{To define a set we can use the roster method, set builder notation, a recursive definition, 
and also we can apply a set operation to other sets. \\

{\bf New! Cartesian product of sets} and {\bf set-wise concatenation of sets of strings}\\


{\bf Definition}: Let $X$ and $Y$ be sets.  The {\bf Cartesian product} of $X$ and $Y$, denoted
$X \times Y$, is the set of all ordered pairs $(x,y)$ where $x \in X$ and $y \in Y$
\[
X \times Y = \{ (x,y) \mid x \in X \text{ and } y \in Y \}
\]
{\bf Definition}: Let $X$ and $Y$ be sets of strings over the same alphabet. The {\bf set-wise concatenation} 
of $X$ and $Y$, denoted $X \circ Y$, is the set of all results of string concatenation $xy$ where $x \in X$ 
and $y \in Y$
\[
X \circ Y = \{ xy \mid x \in X \text{ and } y \in Y \}
\]
}}

{\bf Pro-tip}: the meaning of writing one element next to another like $xy$ depends on the data-types of $x$ and 
$y$. When $x$ and $y$ are strings, the convention is that $xy$ is the result of string concatenation. 
When $x$ and $y$ are numbers, the convention is that $xy$ is the result of multiplication. This is 
(one of the many reasons) why is it very important to declare the data-type of variables before we use them.

{\it Fill in the missing entries in the table}:

\begin{center}
\begin{tabular}{cc}
{\bf  Set} & {\bf Example elements in this set}:\\
\hline 
& \\
$B$ &\A \qquad \C \qquad \G \qquad \U \\
& \\
\hline
& \\
\phantom{$B \times B$} & $(\A, \C)$ \qquad $(\U, \U)$\\
& \\
\hline
& \\
$B \times \{-1,0,1\}$ & \\
& \\
\hline
& \\
$\{-1,0,1\} \times B$ & \\
& \\
\hline
& \\
\phantom{$\{-1,0,1\} \times \{-1,0,1\}  \times \{-1,0,1\} $} & \qquad $(0,0,0)$ \\
& \\
\hline
& \\
$ \{\A, \C, \G, \U \} \circ  \{\A, \C, \G, \U \}$& \\
& \\
\hline
& \\
\phantom{$\{G\} \circ \{G\} \circ \{G\}$} & \qquad $\G\G\G\G$ \\
& \\
\hline

\end{tabular}
\end{center}

\vfill \vfill
\end{document}