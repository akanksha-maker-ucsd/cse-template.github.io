\documentclass[12pt, oneside]{article}

\usepackage[letterpaper, scale=0.89, centering]{geometry}
\usepackage{fancyhdr}
\setlength{\parindent}{0em}
\setlength{\parskip}{1em}

\pagestyle{fancy}
\fancyhf{}
\renewcommand{\headrulewidth}{0pt}
\rfoot{\href{https://creativecommons.org/licenses/by-nc-sa/2.0/}{CC BY-NC-SA 2.0} Version \today~(\thepage)}

\usepackage{amssymb,amsmath,pifont,amsfonts,comment,enumerate,enumitem}
\usepackage{currfile,xstring,hyperref,tabularx,graphicx,wasysym}
\usepackage[labelformat=empty]{caption}
\usepackage[dvipsnames,table]{xcolor}
\usepackage{multicol,multirow,array,listings,tabularx,lastpage,textcomp,booktabs}

\lstnewenvironment{algorithm}[1][] {   
    \lstset{ mathescape=true,
        frame=tB,
        numbers=left, 
        numberstyle=\tiny,
        basicstyle=\rmfamily\scriptsize, 
        keywordstyle=\color{black}\bfseries,
        keywords={,procedure, div, for, to, input, output, return, datatype, function, in, if, else, foreach, while, begin, end, }
        numbers=left,
        xleftmargin=.04\textwidth,
        #1
    }
}
{}
\lstnewenvironment{java}[1][]
{   
    \lstset{
        language=java,
        mathescape=true,
        frame=tB,
        numbers=left, 
        numberstyle=\tiny,
        basicstyle=\ttfamily\scriptsize, 
        keywordstyle=\color{black}\bfseries,
        keywords={, int, double, for, return, if, else, while, }
        numbers=left,
        xleftmargin=.04\textwidth,
        #1
    }
}
{}

\newcommand\abs[1]{\lvert~#1~\rvert}
\newcommand{\st}{\mid}

\newcommand{\A}[0]{\texttt{A}}
\newcommand{\C}[0]{\texttt{C}}
\newcommand{\G}[0]{\texttt{G}}
\newcommand{\U}[0]{\texttt{U}}

\newcommand{\cmark}{\ding{51}}
\newcommand{\xmark}{\ding{55}}

 
\begin{document}
\begin{flushright}
    \StrBefore{\currfilename}{.}
\end{flushright} \section*{Definitions}


\begin{center}
\begin{tabular}{|llp{9.8cm}|}
\hline
{\bf Term} & {\bf Notation Example(s)} & {\bf We say in English \ldots } \\
\hline
sequence & $x_1, \ldots, x_n$ & A sequence $x_1$ to $x_n$ \\
summation & $\sum_{i=1}^n x_i$ or $\displaystyle{\sum_{i=1}^n x_i}$ & The sum of the terms of the sequence $x_1$ to $x_n$ \\
&&\\
all reals & $\mathbb{R}$ & The (set of all) real numbers (numbers on the number line)\\
all integers & $\mathbb{Z}$ & The (set of all) integers (whole numbers including negatives, zero, and positives) \\
all positive integers & $\mathbb{Z}^+$ & The (set of all) strictly positive integers \\
all natural numbers & $\mathbb{N}$ & The (set of all) natural numbers. {\bf Note}: we use the convention that $0$ is a natural number. \\
&&\\
piecewise rule definition & $f(x) = \begin{cases} x & \text{if~}x \geq 0 \\ -x & \text{if~}x<0\end{cases}$ &
Define $f$ of $x$ to be $x$ when $x$ is nonnegative and to be $-x$ when $x$ is negative\\
function application & $f(7)$ & $f$ of $7$ {\bf or} $f$ applied to $7$ {\bf or} the image of $7$ under $f$\\
                     & $f(z)$ & $f$ of $z$ {\bf or} $f$ applied to $z$ {\bf or} the image of $z$ under $f$\\
                     & $f(g(z))$ & $f$ of $g$ of $z$ {\bf or} $f$ applied to the result of $g$ applied to $z$ \\
&&\\
absolute value & $\lvert -3 \rvert$ & The absolute value of $-3$ \\
square root & $\sqrt{9}$ & The non-negative square root of $9$ \\
&&\\


\hline
\end{tabular}
\end{center} \vfill
\section*{Defining sets}


{\it To define sets:}

To define a set using {\bf roster method}, explicitly list its elements. That is,
start with $\{$ then list elements of 
the set separated by commas and close with $\}$.

To define a set using {\bf set builder definition}, either form 
``The set of all $x$ from the universe $U$ such that $x$ is ..." by writing
\[\{x \in U \mid ...x... \}\]
or form ``the collection of all outputs of some operation when the input ranges over the universe $U$"
by writing
\[\{ ...x... \mid x\in U \}\]

We use the symbol $\in$ as ``is an element of'' to indicate membership in a set.\\


{\bf Example sets}: For each of the following, identify whether it's defined using the roster method
or set builder notation and give an example element.
\begin{itemize}
    \item[]$\{ -1, 1\}$\\
    \item[]$\{0, 0 \}$\\
    \item[]$\{-1, 0, 1 \}$\\
    \item[]$\{(x,x,x) \mid x \in \{-1,0,1\} \}$\\
    \item[]$\{ \}$\\
    \item[]$\{ x \in \mathbb{Z} \mid x \geq 0 \}$\\
    \item[]$\{ x \in \mathbb{Z}  \mid x > 0 \}$\\
    \item[]$\{\A,\C,\U,\G\}$ \\
    \item[]$\{\A\U\G, \U\A\G, \U\G\A, \U\A\A \}$\\
\end{itemize}
 \vfill
\section*{Least greatest proofs}


For a set of numbers $X$, how do you formalize ``there is a greatest $X$'' 
or ``there is a least $X$''?

\vspace{30pt}

{\bf Prove} or {\bf  disprove}:  There is a least prime number.

\vspace{100pt}

{\bf Prove} or {\bf  disprove}: There is a greatest integer. 

{\it Approach 1, De Morgan's and universal generalization}: 

\vspace{100pt}

{\it Approach 2, proof by contradiction}: 

\vspace{200pt}

{\it Extra examples}: 
Prove or disprove that $\mathbb{N}$,  $\mathbb{Q}$ each have a
least and a greatest element. 
 \vfill
\section*{Gcd definition}


{\bf Definition}: {\bf Greatest common divisor} Let $a$ and $b$ be integers, not both zero. The largest integer $d$ such that 
$d$ is a  factor of $a$ and $d$ is a factor of  $b$ is called the greatest common divisor of $a$ and $b$ 
and is denoted by $gcd(~(a, b)~)$. \vfill
\section*{Gcd examples}


Why do we restrict to the situation where $a$ and $b$ are not both zero?

\vspace{50pt}


Calculate $gcd(~(10,15)~)$

\vspace{50pt}

Calculate $gcd(~(10,20)~)$

\vspace{50pt} \vfill
\section*{Gcd basic claims}


{\bf Claim}: For any integers $a,b$ (not both zero), $gcd(~(a,b)~) \geq 1$.

{\bf Proof}: {\it Show that $1$ is a common factor of any two integers, so since the gcd 
is the greatest common factor it is greater than or equal to any common factor.}

\vspace{150pt}

{\bf Claim}: For any positive integers $a,b$, $gcd(~(a,b)~) \leq a$ and $gcd( ~(a,b)~) \leq b$.

{\bf Proof} {\it Using the definition of gcd and the fact that factors of a positive integer
are less than or equal to that integer.}

\vspace{150pt}

{\bf Claim}: For any positive integers $a,b$, if $a$ divides $b$ then $gcd(~(a,b)~) = a$.

{\bf Proof} {\it Using previous claim and definition of gcd.}

\vspace{150pt}


{\bf Claim}: For any positive integers $a,b,c$, if there is some integer $q$ such that $a = bq + c$,
\[
    gcd(~(a,b)~) = gcd (~(b,c)~)
\]
{\bf Proof} {\it Prove that any common divisor of $a,b$ divides $c$ and that any common 
divisor of $b,c$ divides $a$.}

\vspace{150pt}
 \vfill
\section*{Gcd lemma relatively prime}


{\bf Lemma}: For any integers $p, q$ (not both zero), 
$gcd \left(~ \left(~\frac{p}{gcd(~(p,q)~)}, \frac{q}{gcd(~(p,q)~)} ~\right) ~\right) = 1$ .
In other words, can reduce to relatively prime integers by dividing by gcd.

{\bf Proof}:

Let $x$ be arbitrary positive integer and assume that $x$ is a 
factor of each of $\frac{p}{gcd(~(p,q)~)}$ and $\frac{q}{gcd(~(p,q)~)}$. 
This gives integers $\alpha$, $\beta$ such that 
\[
    \alpha x = \frac{p}{gcd(~(p,q)~)} \qquad \qquad \beta x = \frac{q}{gcd(~(p,q)~)}
\]
Multiplying both sides by the denominator in the RHS: 
\[
    \alpha x \cdot gcd(~(p,q)~)= p \qquad \qquad \beta x \cdot gcd(~(p,q)~)= q
\]
In other words, $x \cdot gcd(~(p,q)~)$ is a common divisor of $p, q$. By definition of $gcd$, this means
\[
    x \cdot gcd (~(p,q)~) \leq gcd (~(p,q)~)
\]
and since $gcd(~(p,q)~)$ is positive, this means, $x \leq 1$.
\vspace{350pt}
 \vfill
\section*{Least greatest proofs}


For a set of numbers $X$, how do you formalize ``there is a greatest $X$'' 
or ``there is a least $X$''?

\vspace{30pt}

{\bf Prove} or {\bf  disprove}:  There is a least prime number.

\vspace{100pt}

{\bf Prove} or {\bf  disprove}: There is a greatest integer. 

{\it Approach 1, De Morgan's and universal generalization}: 

\vspace{100pt}

{\it Approach 2, proof by contradiction}: 

\vspace{200pt}

{\it Extra examples}: 
Prove or disprove that $\mathbb{N}$,  $\mathbb{Q}$ each have a
least and a greatest element. 
 \vfill
\section*{Gcd definition}


{\bf Definition}: {\bf Greatest common divisor} Let $a$ and $b$ be integers, not both zero. The largest integer $d$ such that 
$d$ is a  factor of $a$ and $d$ is a factor of  $b$ is called the greatest common divisor of $a$ and $b$ 
and is denoted by $gcd(~(a, b)~)$. \vfill
\section*{Gcd examples}


Why do we restrict to the situation where $a$ and $b$ are not both zero?

\vspace{50pt}


Calculate $gcd(~(10,15)~)$

\vspace{50pt}

Calculate $gcd(~(10,20)~)$

\vspace{50pt} \vfill
\section*{Gcd basic claims}


{\bf Claim}: For any integers $a,b$ (not both zero), $gcd(~(a,b)~) \geq 1$.

{\bf Proof}: {\it Show that $1$ is a common factor of any two integers, so since the gcd 
is the greatest common factor it is greater than or equal to any common factor.}

\vspace{150pt}

{\bf Claim}: For any positive integers $a,b$, $gcd(~(a,b)~) \leq a$ and $gcd( ~(a,b)~) \leq b$.

{\bf Proof} {\it Using the definition of gcd and the fact that factors of a positive integer
are less than or equal to that integer.}

\vspace{150pt}

{\bf Claim}: For any positive integers $a,b$, if $a$ divides $b$ then $gcd(~(a,b)~) = a$.

{\bf Proof} {\it Using previous claim and definition of gcd.}

\vspace{150pt}


{\bf Claim}: For any positive integers $a,b,c$, if there is some integer $q$ such that $a = bq + c$,
\[
    gcd(~(a,b)~) = gcd (~(b,c)~)
\]
{\bf Proof} {\it Prove that any common divisor of $a,b$ divides $c$ and that any common 
divisor of $b,c$ divides $a$.}

\vspace{150pt}
 \vfill
\section*{Gcd lemma relatively prime}


{\bf Lemma}: For any integers $p, q$ (not both zero), 
$gcd \left(~ \left(~\frac{p}{gcd(~(p,q)~)}, \frac{q}{gcd(~(p,q)~)} ~\right) ~\right) = 1$ .
In other words, can reduce to relatively prime integers by dividing by gcd.

{\bf Proof}:

Let $x$ be arbitrary positive integer and assume that $x$ is a 
factor of each of $\frac{p}{gcd(~(p,q)~)}$ and $\frac{q}{gcd(~(p,q)~)}$. 
This gives integers $\alpha$, $\beta$ such that 
\[
    \alpha x = \frac{p}{gcd(~(p,q)~)} \qquad \qquad \beta x = \frac{q}{gcd(~(p,q)~)}
\]
Multiplying both sides by the denominator in the RHS: 
\[
    \alpha x \cdot gcd(~(p,q)~)= p \qquad \qquad \beta x \cdot gcd(~(p,q)~)= q
\]
In other words, $x \cdot gcd(~(p,q)~)$ is a common divisor of $p, q$. By definition of $gcd$, this means
\[
    x \cdot gcd (~(p,q)~) \leq gcd (~(p,q)~)
\]
and since $gcd(~(p,q)~)$ is positive, this means, $x \leq 1$.
\vspace{350pt}
 \vfill
\section*{Definitions}


\begin{center}
\begin{tabular}{|llp{9.8cm}|}
\hline
{\bf Term} & {\bf Notation Example(s)} & {\bf We say in English \ldots } \\
\hline
sequence & $x_1, \ldots, x_n$ & A sequence $x_1$ to $x_n$ \\
summation & $\sum_{i=1}^n x_i$ or $\displaystyle{\sum_{i=1}^n x_i}$ & The sum of the terms of the sequence $x_1$ to $x_n$ \\
&&\\
all reals & $\mathbb{R}$ & The (set of all) real numbers (numbers on the number line)\\
all integers & $\mathbb{Z}$ & The (set of all) integers (whole numbers including negatives, zero, and positives) \\
all positive integers & $\mathbb{Z}^+$ & The (set of all) strictly positive integers \\
all natural numbers & $\mathbb{N}$ & The (set of all) natural numbers. {\bf Note}: we use the convention that $0$ is a natural number. \\
&&\\
piecewise rule definition & $f(x) = \begin{cases} x & \text{if~}x \geq 0 \\ -x & \text{if~}x<0\end{cases}$ &
Define $f$ of $x$ to be $x$ when $x$ is nonnegative and to be $-x$ when $x$ is negative\\
function application & $f(7)$ & $f$ of $7$ {\bf or} $f$ applied to $7$ {\bf or} the image of $7$ under $f$\\
                     & $f(z)$ & $f$ of $z$ {\bf or} $f$ applied to $z$ {\bf or} the image of $z$ under $f$\\
                     & $f(g(z))$ & $f$ of $g$ of $z$ {\bf or} $f$ applied to the result of $g$ applied to $z$ \\
&&\\
absolute value & $\lvert -3 \rvert$ & The absolute value of $-3$ \\
square root & $\sqrt{9}$ & The non-negative square root of $9$ \\
&&\\


\hline
\end{tabular}
\end{center} \vfill
\section*{Defining sets}


{\it To define sets:}

To define a set using {\bf roster method}, explicitly list its elements. That is,
start with $\{$ then list elements of 
the set separated by commas and close with $\}$.

To define a set using {\bf set builder definition}, either form 
``The set of all $x$ from the universe $U$ such that $x$ is ..." by writing
\[\{x \in U \mid ...x... \}\]
or form ``the collection of all outputs of some operation when the input ranges over the universe $U$"
by writing
\[\{ ...x... \mid x\in U \}\]

We use the symbol $\in$ as ``is an element of'' to indicate membership in a set.\\


{\bf Example sets}: For each of the following, identify whether it's defined using the roster method
or set builder notation and give an example element.
\begin{itemize}
    \item[]$\{ -1, 1\}$\\
    \item[]$\{0, 0 \}$\\
    \item[]$\{-1, 0, 1 \}$\\
    \item[]$\{(x,x,x) \mid x \in \{-1,0,1\} \}$\\
    \item[]$\{ \}$\\
    \item[]$\{ x \in \mathbb{Z} \mid x \geq 0 \}$\\
    \item[]$\{ x \in \mathbb{Z}  \mid x > 0 \}$\\
    \item[]$\{\A,\C,\U,\G\}$ \\
    \item[]$\{\A\U\G, \U\A\G, \U\G\A, \U\A\A \}$\\
\end{itemize}
 \vfill
\end{document}