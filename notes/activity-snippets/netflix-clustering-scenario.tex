%! app: TODOapp
%! outcome: TODOoutcome

\begin{multicols}{2}
{\bf Scenario}: Good morning! You're a user experience engineer at Netflix. A
product goal is to design customized home pages for groups of users who have
similar interests. You task your team with designing an algorithm for
producing a clustering of users based on their movie interests. Your team
implements two algorithms that produce different clusterings. How do you
decide which one to use? What feedback do you give the team in order to help
them improve? Clearly, you will need to use math.

\columnbreak

{\bf Scenario}: Good morning! You're a user experience engineer at Netflix. A
product goal is to design customized home pages for groups of users who have
similar interests. Your manager tasks you with designing an algorithm for
producing a clustering of users based on their movie interests, with the following constraints:

\end{multicols}

{\bf Idea}: A {\bf clustering} is a partition of the elements in a set with the goal of grouping ``similar'' elements. The definition of ``similar'' can change based on the problem domain.


{\bf Conventions for today}: We will use $U = \{r_1, r_2, \cdots, r_t\}$ to refer to an arbitrary set of user ratings (we'll pick some specific examples to explore) that are a subset of $R_5$. We will be interested in creating partitions $C_1, \cdots, C_m$ of $U$. We'll assume that each user represented by an element of $U$ has a unique ratings tuple.

{\bf Idea}: One way to measure similarity is with functions that measure {\bf distance} between elements.

{\bf Definition}: The distance between two ratings is defined by $d$:

$$d(~ (x_1, \ldots, x_n) , (y_1, \ldots, y_n) ~) =  \displaystyle \sum_{1 \leq i \leq n} | x_i - y_i |$$

Consider $x = (1, 0, 1, 0, 1)$, $y = (1, 1, 1, 0, 1)$, $z = (-1, -1, 0, 0, 0)$, $w = (0, 0, 0, 1, 0)$.

What is $d(x, y)$? $d(x, z)$? $d(z, w)$?

\vfill

\newpage

{\bf Definition}: For a cluster of ratings $C = \{r_1, r_2, \cdots, r_n \} \subseteq U$, the {\bf diameter} of the cluster is defined by:

$$\textit{diameter}(C) = \max_{1 \leq i, j \leq n} (d(r_i, r_j))$$ 

Consider $x = (1, 0, 1, 0, 1)$, $y = (1, 1, 1, 0, 1)$, $z = (-1, -1, 0, 0, 0)$, $w = (0, 0, 0, 1, 0)$.

What is $\textit{diameter}(\{x, y, z\})$? $\textit{diameter}(\{x, y\})$? $\textit{diameter}(\{x, z, w\})$?


\vfill

\textit{diameter} works on single clusters. One way to aggregate across a
clustering $C_1, \cdots, C_m$ is \underline{\hspace{15em}}

Can we easily minimize the sum of diameters?

\vfill

How can we express the idea of {\bf many elements within a small area}? Key idea: ``give credit'' to small diameter clusters with many elements.

\vfill

What is the most useful advice to give the team? Put another way, what is {\bf one number} they can focus on improving, where you (the team leader) understand how that number is calculated.

\vfill



$E_{proj}$
$E_{circ}$