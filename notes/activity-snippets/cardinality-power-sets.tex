%! app: TODOapp
%! outcome: TODOoutcome

{\it Recall}: When $U$ is a set, $\mathcal{P}(U) = \{ X \mid X \subseteq U\}$

{\it Key idea}: For finite sets, the power set of a set has strictly greater size than the set itself.
Does this extend to infinite sets?

{\bf Definition}: For two sets $A, B$, we use the notation $|A| < |B|$ to denote
$(~|A| \leq |B| ~) \land \lnot (~|A| = |B|)$.

\begin{alignat*}{4}
    &\emptyset = \{ \} \qquad &&\mathcal{P}(\emptyset) = \{ \emptyset \} \qquad &&|\emptyset| < |\mathcal{P}(\emptyset)| \\
    &\{1 \} \qquad &&\mathcal{P}(\{1\}) = \{ \emptyset, \{1\} \} \qquad &&|\{1\}| < |\mathcal{P}(\{1\})| \\
    &\{1,2 \} \qquad &&\mathcal{P}(\{1,2\}) = \{ \emptyset, \{1\}, \{2\}, \{1,2\} \} \qquad &&|\{1,2\}| < |\mathcal{P}(\{1,2\})| \\
\end{alignat*}

{\bf $\mathbb{N}$ and its power set}

Example elements of $\mathbb{N}$ 

\vspace{20pt}

Example elements of $\mathcal{P}(\mathbb{N})$

\vspace{20pt}

{\bf Claim}: $| \mathbb{N} | \leq |\mathcal{P} ( \mathbb{N} ) |$

\vspace{100pt}
\newpage
{\bf Claim}: There is an uncountable set.  Example: $\underline{\phantom{~~~\mathcal{P}(\mathbb{N})~~~}}$

{\bf Proof}:  By definition of countable, since $\underline{\phantom{~~~\mathcal{P}(\mathbb{N})~~~}}$
is not finite, {\bf to show} is $|\mathbb{N}| \neq  |\mathcal{P}(\mathbb{N})|$ .

Rewriting using  the definition of  cardinality, {\bf to show} is

\phantom{$\neg \exists f : \mathbb{N} \to \mathcal{P}(\mathbb{N})  ~~(f \text{ is a bijection})~~$}

\phantom{or equivalently $\forall f : \mathbb{N} \to \mathcal{P}(\mathbb{N})  ~~(f \text{ is not a bijection})~~$}


Towards a proof by  universal generalization,  consider  an arbitrary function $f:  \mathbb{N} \to\mathcal{P}(\mathbb{N})$.

{\bf To show}: $f$ is not a bijection.  It's enough to show that $f$ is not onto.

Rewriting using the definition of  onto, {\bf to show}:
\[
\neg  \forall  B \in  \mathcal{P}(\mathbb{N}) ~\exists a \in \mathbb{N}  ~(~f(a) =  B~)
\]
. By logical  equivalence, we can write this as an existential statement:
\[
\underline{\phantom{\qquad\qquad\exists B \in  \mathcal{P}(\mathbb{N}) ~\forall a \in \mathbb{N}  ~(~f(a) \neq  B~)\qquad\qquad}}
\]
In search of a witness, define the following  collection of nonnegative integers:
\[
D_f = \{ n \in \mathbb{N}  ~\mid~  n \notin f(n)  \}
\]
. By  definition  of power  set, since  all elements  of  $D_f$ are  in  $\mathbb{N}$,   $D_f \in \mathcal{P}(\mathbb{N})$.  It's enough to prove the following Lemma: 

{\bf Lemma}: $\forall a \in \mathbb{N}  ~(~f(a) \neq  D_f~)$.


{\bf Proof  of lemma}: \phantom{Towards universal  generalization, consider an arbitrary  $a \in \mathbb{N}$.
By definition  of set equality, {\bf to show} is  $\exists  x ( \neg  (x \in f(a)~  \leftrightarrow  ~x \in D_f))$.
For a witness, consider $x = a$.  There are two cases:  $a \in  f(a)~\vee~a \notin f(a)$. By definition 
of $D_f$, each guarantees that $f(a) \neq  D_f$.}\\

\vspace{50pt}

By  the Lemma, we  have proved that $f$ is not onto, and since $f$ was arbitrary, there are no onto
functions from $\mathbb{N}$ to $\mathcal{P}(\mathbb{N})$. QED


{\bf Where does $D_f$ come from?} The idea is to build a set that would ``disagree" with 
each of the images of $f$ about some element. 

\begin{center}
\begin{tabular}{c|c|ccccccc}
$n \in \mathbb{N}$ & $f(n) = X_n$ &  Is $0   \in X_n$?   & Is $1 \in X_n$?  &  Is $2 \in X_n$?  &  Is $3 \in X_n$?  &
 Is $4 \in X_n$?  &  \ldots & Is $n \in D_f$?\\
\hline
$0$ & $f(0) = X_0$ & {\bf  Y~/~N}  & Y~/~N & Y~/~N & Y~/~N &Y~/~N & \ldots & {\bf  N~/~Y }\\
$1$ & $f(1) = X_1$ & Y~/~N  & {\bf  Y~/~N} & Y~/~N & Y~/~N & Y~/~N & \ldots & {\bf  N~/~Y }\\
$2$ & $f(2) = X_2$ & Y~/~N  & Y~/~N & {\bf  Y~/~N} & Y~/~N &Y~/~N & \ldots & {\bf  N~/~Y }\\
$3$ & $f(3) = X_3$ & Y~/~N  & Y~/~N & Y~/~N & {\bf  Y~/~N} & Y~/~N & \ldots & {\bf  N~/~Y }\\
$4$ & $f(4) = X_4$ & Y~/~N  & Y~/~N & Y~/~N & Y~/~N &{\bf  Y~/~N} & \ldots & {\bf  N~/~Y }\\
\vdots
\end{tabular}
\end{center}