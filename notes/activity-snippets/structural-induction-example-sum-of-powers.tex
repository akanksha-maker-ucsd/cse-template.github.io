%! app: TODOapp
%! outcome: TODOoutcome

The set $\mathbb{N}$ is recursively defined.
Therefore, the function $sumPow: \mathbb{N} \to \mathbb{N}$
which computes, for input $i$, the sum of the first $i$ powers of $2$ is defined
recursively by

\begin{alignat*}{2}
    \text{Basis step:  } \qquad & sumPow(0) = 1 &\\
    \text{Recursive step:  } & \text{If } x \in \mathbb{N} \text{, then } &sumPow(x+1) = sumPow(x) + 2^{x+1}
\end{alignat*}

$sumPow(0) =$

\vspace{20pt}

$sumPow(1) =$

\vspace{20pt}

$sumPow(2) =$

\vspace{20pt}


Fill in the blanks in the following proof of 
\[
    \forall n \in \mathbb{N}~(sumPow(n) = 2^{n+1} - 1)
\]

{\bf Proof}: Since $\mathbb{N}$ is recursively defined, we proceed by \underline{\phantom{structural induction \hspace{0.3in}}}.

{\bf Basis case}: We need to show that \underline{\phantom{$sumPow(0) = 2^{0+1} - 1$ \hspace{0.2in}}}.
Evaluating each side: $LHS = sumPow(0) = 1$ by the basis case in the recursive definition
of $sumPow$; $RHS = 2^{0+1} - 1 = 2^1 - 1 = 2-1 = 1$. Since $1=1$, the equality holds.

{\bf Recursive step} Consider arbitrary natural number $n$ and assume, as the 
\underline{\phantom{Induction Hypothesis (IH)}} that $sumPow(n) = 2^{n+1} - 1$. We need to show that
\underline{\phantom{$sumPow(n+1) = 2^{(n+1) + 1} - 1$}}.  Evaluating each side: 
\[
LHS = sumPow(n+1) \overset{\text{rec def}}{=} sumPow(n)  + 2^{n+1}\overset{\text{IH}}{=} (2^{n+1} - 1) + 2^{n+1}.
\]
\[
RHS = 2^{(n+1)+1}- 1 \overset{\text{exponent rules}}{=} 2 \cdot 2^{n+1} -1  = \left(2^{n+1} + 2^{n+1} \right) - 1
\overset{\text{regrouping}}{=}  (2^{n+1} - 1) + 2^{n+1} 
\]
Thus, $LHS = RHS$. The structural induction is complete and we have proved the universal generalization.
$\square$


{\it Challenge:} Connect the function $sumPow$ to binary expansions of positive integers.
