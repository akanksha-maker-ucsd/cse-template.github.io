\documentclass[12pt, oneside]{article}

\usepackage[letterpaper, scale=0.89, centering]{geometry}
\usepackage{fancyhdr}
\setlength{\parindent}{0em}
\setlength{\parskip}{1em}

\pagestyle{fancy}
\fancyhf{}
\renewcommand{\headrulewidth}{0pt}
\rfoot{\href{https://creativecommons.org/licenses/by-nc-sa/2.0/}{CC BY-NC-SA 2.0} Version \today~(\thepage)}

\usepackage{amssymb,amsmath,pifont,amsfonts,comment,enumerate,enumitem}
\usepackage{currfile,xstring,hyperref,tabularx,graphicx,wasysym}
\usepackage[labelformat=empty]{caption}
\usepackage[dvipsnames,table]{xcolor}
\usepackage{multicol,multirow,array,listings,tabularx,lastpage,textcomp,booktabs}

\lstnewenvironment{algorithm}[1][] {   
    \lstset{ mathescape=true,
        frame=tB,
        numbers=left, 
        numberstyle=\tiny,
        basicstyle=\rmfamily\scriptsize, 
        keywordstyle=\color{black}\bfseries,
        keywords={,procedure, div, for, to, input, output, return, datatype, function, in, if, else, foreach, while, begin, end, }
        numbers=left,
        xleftmargin=.04\textwidth,
        #1
    }
}
{}
\lstnewenvironment{java}[1][]
{   
    \lstset{
        language=java,
        mathescape=true,
        frame=tB,
        numbers=left, 
        numberstyle=\tiny,
        basicstyle=\ttfamily\scriptsize, 
        keywordstyle=\color{black}\bfseries,
        keywords={, int, double, for, return, if, else, while, }
        numbers=left,
        xleftmargin=.04\textwidth,
        #1
    }
}
{}

\newcommand\abs[1]{\lvert~#1~\rvert}
\newcommand{\st}{\mid}

\newcommand{\A}[0]{\texttt{A}}
\newcommand{\C}[0]{\texttt{C}}
\newcommand{\G}[0]{\texttt{G}}
\newcommand{\U}[0]{\texttt{U}}

\newcommand{\cmark}{\ding{51}}
\newcommand{\xmark}{\ding{55}}

 
\begin{document}
\begin{flushright}
    \StrBefore{\currfilename}{.}
\end{flushright} \section*{Defining sets}


{\it To define sets:}

To define a set using {\bf roster method}, explicitly list its elements. That is,
start with $\{$ then list elements of 
the set separated by commas and close with $\}$.

To define a set using {\bf set builder definition}, either form 
``The set of all $x$ from the universe $U$ such that $x$ is ..." by writing
\[\{x \in U \mid ...x... \}\]
or form ``the collection of all outputs of some operation when the input ranges over the universe $U$"
by writing
\[\{ ...x... \mid x\in U \}\]

We use the symbol $\in$ as ``is an element of'' to indicate membership in a set.\\


{\bf Example sets}: For each of the following, identify whether it's defined using the roster method
or set builder notation and give an example element.
\begin{itemize}
    \item[]$\{ -1, 1\}$\\
    \item[]$\{0, 0 \}$\\
    \item[]$\{-1, 0, 1 \}$\\
    \item[]$\{(x,x,x) \mid x \in \{-1,0,1\} \}$\\
    \item[]$\{ \}$\\
    \item[]$\{ x \in \mathbb{Z} \mid x \geq 0 \}$\\
    \item[]$\{ x \in \mathbb{Z}  \mid x > 0 \}$\\
    \item[]$\{\A,\C,\U,\G\}$ \\
    \item[]$\{\A\U\G, \U\A\G, \U\G\A, \U\A\A \}$\\
\end{itemize}
 \vfill
\section*{Rna motivation}


RNA is made up of strands of four different bases that encode genomic information
in specific ways.\\
The bases are elements of the set 
$B  = \{\A, \C, \U, \G \}$.


Formally, to define the set of all RNA strands, we need more than roster
method or set builder descriptions. 

 \vfill
\section*{Set recursive examples}


{\bf Definition} The set of nonnegative integers $\mathbb{N}$ is defined (recursively) by: 
\[
\begin{array}{ll}
\textrm{Basis Step: } & \phantom{0 \in \mathbb{N}} \\
\textrm{Recursive Step: } & \phantom{\textrm{If } n \in \mathbb{N} \textrm{, then } n+1 \in \mathbb{N}}
\end{array}
\]

Examples: 

{\bf Definition} The set of all integers $\mathbb{Z}$ is defined (recursively) by: 
\[
\begin{array}{ll}
\textrm{Basis Step: } & \phantom{0 \in \mathbb{Z}} \\
\textrm{Recursive Step: } & \phantom{\textrm{If } x \in \mathbb{Z} \textrm{, then } x+1 \in \mathbb{Z}
\textrm{ and } x-1 \in \mathbb{Z}}
\end{array}
\]

Examples: 

\vfill

{\bf Definition} The set of RNA strands $S$ is defined (recursively) by:
\[
\begin{array}{ll}
\textrm{Basis Step: } & \A \in S, \C \in S, \U \in S, \G \in S \\
\textrm{Recursive Step: } & \textrm{If } s \in S\textrm{ and }b \in B \textrm{, then }sb \in S
\end{array}
\]
where $sb$ is string concatenation.

Examples: 

\vfill

{\bf Definition} The set of bitstrings (strings of 0s and 1s) is defined (recursively) by:
\[
\begin{array}{ll}
\textrm{Basis Step: } & \phantom{\lambda \in X} \\
\textrm{Recursive Step: } & \phantom{\textrm{If } s \in X \textrm{, then } s0 \in X \text{ and } s1 \in X}
\end{array}
\]

{\it Notation:} We call the set of bitstrings $\{0,1\}^*$.

Examples: 

\vfill \vfill
\end{document}