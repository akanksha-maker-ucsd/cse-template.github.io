\documentclass[12pt, oneside]{article}

\usepackage[letterpaper, scale=0.8, centering]{geometry}
\usepackage{fancyhdr}
\setlength{\parindent}{0em}
\setlength{\parskip}{1em}

\pagestyle{fancy}
\fancyhf{}
\renewcommand{\headrulewidth}{0pt}
\rfoot{{\footnotesize Copyright Mia Minnes, 2021, Version \today~(\thepage)}}

\author{CSE20F21}

\newcommand{\instructions}{{\bf For all HW assignments:}

Weekly homework may be done individually or in groups of up to 3 students. 
You may switch HW partners for different HW assignments. 
The lowest HW score will not be included in your overall HW average. 
Please ensure your name(s) and PID(s) are clearly visible on the first page of your homework submission.

All submitted homework for this class must be typed. 
Diagrams may be hand-drawn and scanned and included in the typed document. 
You can use a word processing editor if you like (Microsoft Word, Open Office, Notepad, Vim, Google Docs, etc.) 
but you might find it useful to take this opportunity to learn LaTeX. 
LaTeX is a markup language used widely in computer science and mathematics. 
The homework assignments are typed using LaTeX and you can use the source files 
as templates for typesetting your solutions\footnote{To use this template, copy the source file (extension \texttt{.tex}) 
to your working directory or upload to Overleaf.}.


{\bf Integrity reminders}
\begin{itemize}
\item Problems should be solved together, not divided up between the partners. The homework is
designed to give you practice with the main concepts and techniques of the course, 
while getting to know and learn from your classmates.
\item You may not collaborate on homework with anyone other than your group members.
You may ask questions about the homework in office hours (of the instructor, TAs, and/or tutors) and 
on Piazza (as private notes viewable only to the Instructors).  
You \emph{cannot} use any online resources about the course content other than the class material 
from this quarter -- this is primarily to ensure that we all use consistent notation and
definitions we will use this quarter.
\item Do not share written solutions or partial solutions for homework with 
other students in the class who are not in your group. Doing so would dilute their learning 
experience and detract from their success in the class.
\end{itemize}

}
\usepackage{amssymb,amsmath,pifont,amsfonts,comment,enumerate,enumitem}
\usepackage{currfile,xstring,hyperref,tabularx,graphicx,wasysym}
\usepackage[labelformat=empty]{caption}
\usepackage[dvipsnames,table]{xcolor}
\usepackage{multicol,multirow,array,listings,tabularx,lastpage,textcomp,booktabs}

\lstnewenvironment{algorithm}[1][] {   
    \lstset{ mathescape=true,
        frame=tB,
        numbers=left, 
        numberstyle=\tiny,
        basicstyle=\rmfamily\scriptsize, 
        keywordstyle=\color{black}\bfseries,
        keywords={,procedure, div, for, to, input, output, return, datatype, function, in, if, else, foreach, while, begin, end, }
        numbers=left,
        xleftmargin=.04\textwidth,
        #1
    }
}
{}
\lstnewenvironment{java}[1][]
{   
    \lstset{
        language=java,
        mathescape=true,
        frame=tB,
        numbers=left, 
        numberstyle=\tiny,
        basicstyle=\ttfamily\scriptsize, 
        keywordstyle=\color{black}\bfseries,
        keywords={, int, double, for, return, if, else, while, }
        numbers=left,
        xleftmargin=.04\textwidth,
        #1
    }
}
{}

\newcommand\abs[1]{\lvert~#1~\rvert}
\newcommand{\st}{\mid}

\newcommand{\A}[0]{\texttt{A}}
\newcommand{\C}[0]{\texttt{C}}
\newcommand{\G}[0]{\texttt{G}}
\newcommand{\U}[0]{\texttt{U}}

\newcommand{\cmark}{\ding{51}}
\newcommand{\xmark}{\ding{55}}

 
 
\title{HW4 Proofs and Sets}
\date{Due: Tuesday, November 2, 2021 at 11:00PM on Gradescope}

\begin{document}
\maketitle
\thispagestyle{fancy}

{\bf In this assignment,}

You will analyze statements and determine if they are true or false using valid proof strategies.
You will also determine if candidate arguments are valid.

Instructions and academic integrity reminders for all homework assignments in 
CSE20 this quarter are on the class website and on the hw1-definitions-and-notations
assignment.

You will submit this assignment via Gradescope
(\href{https://www.gradescope.com}{https://www.gradescope.com}) 
in the assignment called ``hw4-proofs-and-sets''.

{\bf Resources}: To review the topics you are working with 
for this assignment, see the class material from Week 5.
We will post frequently asked questions and our answers to them in a 
pinned Piazza post.


\newpage
In your proofs and disproofs of statements below, justify each  step
by reference to  a component of the  following proof  strategies
we  have discussed so far, and/or to relevant definitions and calculations.
\begin{itemize}
    \item A counterexample can be used to prove that  $\forall x P(x)$ is {\bf false}.
    \item  A witness can be used to prove that  $\exists x P(x)$ is {\bf true}.
    \item {\bf Proof of universal by exhaustion}: To prove that $\forall x \, P(x)$
is true when $P$ has a finite domain, evaluate the predicate at {\bf each} domain element to confirm that it is always T.
    \item  {\bf Proof by universal generalization}: To prove that $\forall x \, P(x)$
is true, we can take an arbitrary element $e$ from the domain and show that $P(e)$ is true, without making any assumptions 
about $e$ other than that it comes from the domain.
    \item To  prove  that $\exists x P(x)$ is {\bf false}, write the universal statement that is 
    logically equivalent to its negation and then prove it true using universal generalization.
    \item {\bf Strategies for conjunction}: To prove that $p \land q$ is true, have two subgoals: 
    subgoal (1) prove $p$ 
is  true; and, subgoal (2) prove $q$ is true. To prove that $p \land q$ is false, it's enough to prove that $p$ is false.
 To prove that $p \land q$ is false, it's enough to prove that $q$ is false.
    \item {\bf Proof of Conditional by Direct Proof}: To prove that the implication $p \to q$ is true, 
    we can assume $p$ is true and use that assumption to show $q$ is true.
    \item {\bf Proof of Conditional by Contrapositive Proof}: To prove that the implication $p \to q$ is true, 
    we can assume $\neg q$ is true and use that assumption to show $\neg p$ is true.
    \item {\bf Proof of disjuction using equivalent conditional}: To prove that the 
    disjunction $p \lor q$ is true, we can rewrite it equivalently as $\lnot p \to q$ and
    then use direct proof or contrapositive proof.
    \item {\bf Proof by Cases}: To prove $q$ when we know $p_1 \lor p_2$, show that $p_1 \to q$ and $p_2 \to q$.
\end{itemize}

{\bf Assigned questions}

\begin{enumerate}
   \item Consider the predicate $Pr(x)$ over the set of integers, which evaluates to $T$ exactly when 
   $x$ is prime. Consider the following statements.
   
    \begin{multicols}{2}
    \begin{enumerate}[label=(\roman*)]
        \item $\exists x \in \mathbb{Z}~ \forall y \in \mathbb{Z}~(~x \leq y \to Pr(y)~)$
        \item $\exists x \in \mathbb{Z}~ \forall y \in \mathbb{Z}~(~y \leq x \to Pr(y)~)$
        \item $\forall x \in \mathbb{Z}~ \exists y \in \mathbb{Z}~(~x \leq y \to Pr(y)~)$
        \item $\forall x \in \mathbb{Z}~ \exists y \in \mathbb{Z}~(~y \leq x \to Pr(y)~)$
        \item $\exists x \in \mathbb{Z}~ \forall y \in \mathbb{Z}~(~Pr(y) \to y \leq x~)$
        \item $\exists x \in \mathbb{Z}~ \forall y \in \mathbb{Z}~(~Pr(y) \to x \leq y~)$
        \item $\forall x \in \mathbb{Z}~ \exists y \in \mathbb{Z}~(~Pr(y) \to y \leq x~)$
        \item $\forall x \in \mathbb{Z}~ \exists y \in \mathbb{Z}~(~Pr(y) \to x \leq y~)$
    \end{enumerate}
    \end{multicols}
   
   \begin{enumerate}
   
   \item ({\it Graded for correctness of choice and fair effort completeness in justification
   \footnote{Graded for correctness means your solution will be
   evaluated not only on the correctness of your answers, but on your ability to 
   present your ideas clearly and logically. You should explain how you arrived at your conclusions, using 
   mathematically sound reasoning. Whether you use formal proof techniques or write a more informal argument for why 
   something is true, your answers should always be well-supported. Your goal should be to convince the reader that 
   your results and methods are sound. Graded for fair effort completeness means 
   you will get full credit so long as your submission demonstrates honest 
   effort to answer the question. You will not be penalized for incorrect answers.}}) 
   Which of the statements (i) - (viii) is being {\bf proved} by the following proof:
   \begin{quote}
     Choose $x = 1$, an integer, and we will work to show
     it is a {\bf witness} for the existential claim. By universal generalization, {\bf choose} $e$ to be an {\bf arbitrary} integer. 
     Towards a {\bf direct proof}, {\bf assume} that $Pr(e)$ holds. We {\bf WTS} that $1 \leq e$.
     By definition of the  predicate $Pr$, since $Pr(e)$ is true, $e > 1$. By definition of $\leq$, 
     this means that $1 \leq e$, as required and the claim has been proved. $\square$
   \end{quote}
   
   
   {\it Hint: it may be useful to 
   identify the key words in the proof that indicate proof strategies.}
   
   \item ({\it Graded for correctness of choice and fair effort completeness in justification}) 
   Which of the statements (i) - (viii) is being {\bf disproved} by the following proof:
   \begin{quote}
     To disprove the statement, we will prove the universal
     statement that is logically equivalent to its negation. 
     By universal generalization, {\bf choose} $e$ to be an {\bf arbitrary} integer. 
     We need to find a {\bf witness} integer $y$ such that $y \leq e$ and $\lnot Pr(y)$.
     Notice that $e > 1 \lor e \leq 1$ is true, and we proceed in a {\bf proof by cases}.
     {\bf Case 1}: Assume $e > 1$ and {\bf WTS} there is a witness integer $y$ such that
     $y \leq e$ and $\lnot Pr(y)$. Choose $y = 0$, an integer. Then, since by {\bf case assumption}
     $1 < e$, we have $y = 0 \leq 1 \leq e$.
     Moreover, since $y = 0$, $y > 1$ is false and so (by the definition of $Pr$), the predicate $Pr$
     evaluated at $y$ is false, as required to prove the {\bf conjunction} $y \leq e$ and $\lnot Pr(y)$. 
     {\bf Case 2}: 
     Assume $e \leq 1$ and {\bf WTS} there is a witness integer $y$ such that
     $y \leq e$ and $\lnot Pr(y)$. Choose $y = e-1$, an integer (because subtracting
     $1$ from the integer $e$ still gives an integer). By definition of subtraction, $y = e-1 \leq e$.
     Moreover, since by the {\bf case assumption} $y = e-1 \leq 1-1= 0$, $y > 1$ is false. Thus, 
    (by the definition of $Pr$), the predicate $Pr$
     evaluated at $y$ is false. We have proved the {\bf conjunction} $y \leq e$ and $\lnot Pr(y)$ as required.
     Since each case is complete, the proof by cases is complete and the original
     statement has been disproved.  $\square$
   \end{quote}
   
   {\it Hint: it may be useful to 
   identify the key words in the proof that indicate proof strategies.}

   \item ({\it Graded for correctness of evaluation of statement (is it true or false?)
   and fair effort completeness of the translation and of the proof}) 
    Translate the statement to English and then prove or disprove it
   $$\forall x \in \mathbb{Z}~ \forall y \in \mathbb{Z}~(~x \neq y \to (Pr(x) \lor Pr(y))~)$$

   \item ({\it Graded for correctness of evaluation of statement (is it true or false?) 
   and fair effort completeness of the translation and proof}) 
   Translate the statement to English and then prove or disprove it
   $$\left( ~\forall x \in \mathbb{Z} ~Pr(x)~\right) \oplus \left(~\exists x \in \mathbb{Z} ~Pr(x) ~\right)$$

   \item ({\it Graded for correctness of evaluation of statement (is it true or false?) 
   and fair effort completeness of the translation and of the proof}) 
    Translate the statement to English and then prove or disprove it
   $$\forall x \in \mathbb{Z}~ \forall y \in \mathbb{Z}~(~(~Pr(x) \land Pr(y)~) \leftrightarrow Pr(x+y)~)$$
   
   \item ({\it Graded for correctness of evaluation of statement (is it true or false?)
   and fair effort completeness of the translation and of the proof}) 
    Translate the statement to English and then prove or disprove it
   $$\forall x \in \mathbb{Z}~ (~Pr(x) \to \exists y \in \mathbb{Z}~(~x < y \land Pr(y)~)$$

   \end{enumerate}

   
   \item Let $W = \mathcal{P}(\{1,2,3,4,5\})$. 
   
   \rule{0.5\textwidth}{.4pt}
   
   {\it Sample response that can be used as reference for the detail expected 
   in your answers for this question:} 
   
   To give a witness for the existential claim
   $$ \exists B \in W~( B \in ~\{ X \in W ~|~ 1 \in X \} \cap \{ X \in W ~|~  2 \in X \}~~)$$
   consider $B = \{ 1,2\}$. To prove that this is a valid witness, we need
   to show that it is in the domain of quantification $W$ and that 
   it makes the predicate being quantified evaluate to true. By definition 
   of set-builder notation and intersection, it's enough to prove
   that $\{1,2\} \in W$ and that $1 \in \{1,2\}$ and that $2 \in \{1,2\}$.
   \begin{itemize}
   \item By definition of power set, elements of $W$ are subsets of $\{1,2,3,4,5\}$. Since
   each element in $\{1,2\}$ is an element of $\{1,2,3,4,5\}$, $\{1,2\}$ is a subset of $\{1,2,3,4,5\}$ 
   and hence is an element of $W$. 
   \item Also, by definition of the roster method, $1 \in \{1,2\}$. 
   \item Similarly, by definition of roster method, $2 \in \{1,2\}$.
   \end{itemize}
   Thus $B = \{1,2\}$ is an element of the domain which is in the intersection of the 
   two sets mentioned in the predicate being quantified and is a witness to the existential claim. QED
   
   \rule{0.5\textwidth}{.4pt}
   
   
   \begin{enumerate}
   \item ({\it Graded for correctness}) Give a witness to the existential claim
   $$ \exists X \in W ~(~X \cup X = \emptyset~)$$
   Justify your example by explanations that include references to the relevant definitions.
   
   \item ({\it Graded for correctness}) Give a counterexample to the universal claim
   $$ \forall X \in W ~( \{ a \in X \mid a \textrm{ is even} \} \subsetneq X~)$$
   Justify your example by explanations 
   that include references to the relevant definitions.
   
   \item  ({\it Graded for correctness}) Give a witness to the existential claim
   $$ \exists (X,Y) \in W \times W ~(~X \cup Y = Y~)$$
   Justify your example by explanations that include references to the relevant definitions.
   \end{enumerate}
   

   \item Recall our representation of movie preferences in a three-movie database 
   using $1$ in a component to indicate liking the movie represented by that component, 
   $-1$ to indicate not liking the movie, and $0$ to indicate neutral opinion or
   haven't seen the movie. We call $Rt$ the set of all ratings $3$-tuples. 
   We defined the function 
   $d_0: Rt\times Rt \to \mathbb{R}$ which takes an ordered pair of ratings $3$-tuples and returns a measure
   of the distance between them 
   given by
   \[
   d_0 (~(~ (x_1, x_2, x_3), (y_1, y_2, y_3) ~) ~) = \sqrt{ (x_1 - y_1)^2 + (x_2 - y_2)^2 + (x_3 -y_3)^2}
   \]
   Another measure of the distance between a pair of ratings $3$-tuples is given by 
   the following function $d_1: Rt\times Rt \to \mathbb{R}$ given by 
   \[
   d_1 (~(~ (x_1, x_2, x_3), (y_1, y_2, y_3) ~) ~) = \sum_{i=1}^3 |x_i - y_i|
   \]
   \begin{enumerate}
    \item    For each of the statements below, first translate them symbolically (using
        quantifiers, logical operators, and arithmetic operations), then determine whether each 
        is true or false by applying the proof strategies to prove each statement or its negation.
        ({\it Graded for correctness of evaluation of statement (is it true or false?) and 
        fair effort completeness of the translation and of the proof}) 
        \begin{enumerate}
            \item For all ordered pairs of ratings $3$-tuples, the value of the function $d_0$ 
            is greater than the value of the function $d_1$.
            \item The maximum value of the function $d_1$ is greater than the maximum value of the function $d_0$.
        \end{enumerate}

    \item ({\it Graded for correctness}) Write a statement about 3-tuples of movie ratings that uses the function 
    $d_1$ and has at least one universal and one existential quantifier. Your response will be 
    graded correct if all the syntax in your statement is correct.

    \item ({\it Graded for fair effort completeness}) Translate the property you wrote symbolically in the 
    last step to English. Indicate if it is true, false, or if you don't know 
    (sometimes we can write interesting properties, and we're not sure if they are true or not!). 
    Give informal justification for whether  you think it is true/ false, or explain why 
    the proof strategies we have so far do not appear to  be sufficient to determine whether the statement holds.
    \end{enumerate}

\end{enumerate}
\end{document}